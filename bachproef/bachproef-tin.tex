%===============================================================================
% LaTeX sjabloon voor de bachelorproef toegepaste informatica aan HOGENT
% Meer info op https://github.com/HoGentTIN/bachproef-latex-sjabloon
%===============================================================================

\documentclass{bachproef-tin}

\usepackage{hogent-thesis-titlepage} % Titelpagina conform aan HOGENT huisstijl
\usepackage{listings}
\usepackage{xcolor}


\definecolor{delim}{RGB}{20,105,176}
\definecolor{numb}{RGB}{106, 109, 32}
\definecolor{string}{rgb}{0.64,0.08,0.08}

\lstdefinelanguage{json}{
	numbers=left,
	numberstyle=\small,
	frame=single,
	rulecolor=\color{black},
	showspaces=false,
	showtabs=false,
	breaklines=true,
	postbreak=\raisebox{0ex}[0ex][0ex]{\ensuremath{\color{gray}\hookrightarrow\space}},
	breakatwhitespace=true,
	basicstyle=\ttfamily\small,
	upquote=true,
	morestring=[b]",
	stringstyle=\color{string},
	literate=
	*{0}{{{\color{numb}0}}}{1}
	{1}{{{\color{numb}1}}}{1}
	{2}{{{\color{numb}2}}}{1}
	{3}{{{\color{numb}3}}}{1}
	{4}{{{\color{numb}4}}}{1}
	{5}{{{\color{numb}5}}}{1}
	{6}{{{\color{numb}6}}}{1}
	{7}{{{\color{numb}7}}}{1}
	{8}{{{\color{numb}8}}}{1}
	{9}{{{\color{numb}9}}}{1}
	{\{}{{{\color{delim}{\{}}}}{1}
	{\}}{{{\color{delim}{\}}}}}{1}
	{[}{{{\color{delim}{[}}}}{1}
	{]}{{{\color{delim}{]}}}}{1},
}

%%---------- Documenteigenschappen ---------------------------------------------
% TODO: Vul dit aan met je eigen info:

% De titel van het rapport/bachelorproef
\title{Personalized Search: Graph vs. OLAP Database}

% Je eigen naam
\author{Shauni Van de Velde}

% De naam van je promotor (lector van de opleiding)
\promotor{Guy Dekoning}

% De naam van je co-promotor. Als je promotor ook je opdrachtgever is en je
% dus ook inhoudelijk begeleidt (en enkel dan!), mag je dit leeg laten.
\copromotor{Nicolas Lierman}

% Indien je bachelorproef in opdracht van/in samenwerking met een bedrijf of
% externe organisatie geschreven is, geef je hier de naam. Zoniet laat je dit
% zoals het is.
\instelling{MultiMinds}

% Academiejaar
\academiejaar{2019-2020}

% Examenperiode
%  - 1e semester = 1e examenperiode => 1
%  - 2e semester = 2e examenperiode => 2
%  - tweede zit  = 3e examenperiode => 3
\examenperiode{2}

%===============================================================================
% Inhoud document
%===============================================================================

\begin{document}
\raggedbottom
	

%---------- Taalselectie -------------------------------------------------------
% Als je je bachelorproef in het Engels schrijft, haal dan onderstaande regel
% uit commentaar. Let op: de tekst op de voorkaft blijft in het Nederlands, en
% dat is ook de bedoeling!

%\selectlanguage{english}

%---------- Titelblad ----------------------------------------------------------
\inserttitlepage

%---------- Samenvatting, voorwoord --------------------------------------------
\usechapterimagefalse
%%=============================================================================
%% Voorwoord
%%=============================================================================

\chapter*{\IfLanguageName{dutch}{Woord vooraf}{Preface}}
\label{ch:voorwoord}

%% TODO:
%% Het voorwoord is het enige deel van de bachelorproef waar je vanuit je
%% eigen standpunt (``ik-vorm'') mag schrijven. Je kan hier bv. motiveren
%% waarom jij het onderwerp wil bespreken.
%% Vergeet ook niet te bedanken wie je geholpen/gesteund/... heeft

Deze bachelorproef met als titel "Personalized Search: Neo4j vs Elasticsearch" werd geschreven in het kader van het behalen van een diploma als bachelor in de Toegepaste Informatica aan de Hogeschool Gent.

Dit onderwerp werd mij voorgesteld door MultiMinds, en sprak mij aan omdat dit ondanks de recente wetgeving rond GDPR nog steeds een actueel onderwerp is. Ik had reeds een idee in mijn hoofd  van hoe personalisatie in e-commerce toepassingen in zijn werk ging, en ik wou mij hier met plezier in verdiepen. Dit onderzoek was een leerzame ervaring voor mij, en als ik erop terugblik zou ik geen ander onderwerp gekozen hebben. 

Bij deze zou ik graag nog even enkele personen bedanken die mij geholpen en gesteund hebben bij het maken van deze bachelorproef, zonder hun zou dit onderzoek niet tot stand kunnen gekomen zijn.

Als eerste zou ik graag mijn copromotor, Nicolas Lierman van MultiMinds, bedanken voor het aanbieden van dit interessante onderwerp, alsook de technische ondersteuning en het nazien van dit werk.

Ten tweede zou ik ook graag mijn promotor, Guy Dekoning van Hogeschool Gent, bedanken voor de vlotte en zeer duidelijke feedback, en het blijven steun tonen in mijn onderzoek. 

Ten slotte wil ik ook graag mijn ouders bedanken, om mij in alle rust te laten werken ondanks dat ik hele dagen thuiszit; mijn medestudenten, die mijn rubberen eendje wouden spelen; en mijn vrienden, in het bijzonder Angelo Carly, om mij morele steun te bieden en inhoudelijke feedback te geven van het perspectief van een buitenstaander.



%%=============================================================================
%% Samenvatting
%%=============================================================================

% TODO: De "abstract" of samenvatting is een kernachtige (~ 1 blz. voor een
% thesis) synthese van het document.
%
% Deze aspecten moeten zeker aan bod komen:
% - Context: waarom is dit werk belangrijk?
% - Nood: waarom moest dit onderzocht worden?
% - Taak: wat heb je precies gedaan?
% - Object: wat staat in dit document geschreven?
% - Resultaat: wat was het resultaat?
% - Conclusie: wat is/zijn de belangrijkste conclusie(s)?
% - Perspectief: blijven er nog vragen open die in de toekomst nog kunnen
%    onderzocht worden? Wat is een mogelijk vervolg voor jouw onderzoek?
%
% LET OP! Een samenvatting is GEEN voorwoord!

%%---------- Nederlandse samenvatting -----------------------------------------
%
% TODO: Als je je bachelorproef in het Engels schrijft, moet je eerst een
% Nederlandse samenvatting invoegen. Haal daarvoor onderstaande code uit
% commentaar.
% Wie zijn bachelorproef in het Nederlands schrijft, kan dit negeren, de inhoud
% wordt niet in het document ingevoegd.

\IfLanguageName{english}{%
\selectlanguage{dutch}
\chapter*{Samenvatting}
\lipsum[1-4]
\selectlanguage{english}
}{}

%%---------- Samenvatting -----------------------------------------------------
% De samenvatting in de hoofdtaal van het document

\chapter*{\IfLanguageName{dutch}{Samenvatting}{Abstract}}

In samenwerking met MultiMinds werd besloten onderzoek te voeren naar de haalbaarheid van het gebruik van twee vooraf bepaalde opties voor het implementeren van een gepersonaliseerde zoekfunctie bij e-commerce toepassingen, de  opties die gekozen werden door de opdrachtgever zijn Neo4j en Elasticsearch.  

De reden voor dit onderzoek is dat zo goed als alle gekende webshops gebruik maken van een zoekfunctie, maar dat deze vaak beperkt blijft tot de productcatalogus. In dit onderzoek wordt dus nagegaan of het mogelijk is deze zoekfunctie uit te breiden met data van gebruikers om zo een meer gepersonaliseerd resultaat te kunnen opleveren.

Enkele voorbeelden van dergelijke gebruikersdata zijn geslacht, leeftijd, locatie, etc., maar ook bijvoorbeeld familiale verbanden. Zo is het bijvoorbeeld wenselijk dat in het geval van twee samenwonenden, Persoon A en persoon B, er merken worden aanbevolen aan Persoon A waarvan Persoon B regelmatig producten koopt.

In eerste instantie werd er een literatuurstudie gedaan naar de reeds bestaande vormen van personalisatie, al dan niet binnen e-commerce toepassingen als marketingvorm.

Ten tweede werden de verschillende types van aanbevelingssystemen onderzocht. Daarbij werden twee belangrijke families gevonden: Collaborative Filtering-algoritmen, en Content Based-algoritmen. Voor de implementatie van dergelijke algoritmen zijn ook hybride vormen mogelijk, deze hybride vormen werden gebruikt in dit onderzoek.

Ten derde werd er ook een literatuuronderzoek gedaan naar de twee platformen, Neo4j en Elasticsearch. In dit deel van het onderzoek werd bekeken of deze platformen de mogelijkheden hadden om een aanbevelingssysteem te implementeren dat voldoende accurate resultaten zou opleveren. Uit dit onderzoek bleek dat Neo4j sterk uitblinkt in het gebruiken van data over de relaties tussen twee personen, waar Elasticsearch eerder uitblinkt in het zeer snel opleveren van resultaten die aan bepaalde criteria voldoen.

Als vierde puntje in het onderzoek werd ook verder ingegaan op een nadeel dat het personaliseren van e-commerce toepassingen met zich mee kan brengen, namelijk een 'Filter Bubble', waarbij gebruikers enkel nog resultaten te zien krijgen die volledig passen bij hun profiel. Bijgevolg zullen gebruikers weinig of zelfs geen resultaten mogen verwachten die door hun nog niet gezien werden.

Als laatste deel van de literatuurstudie werd bekeken of de implementatie van een systeem dat op deze manier gebruik maakt van persoonlijke data en gegevens wel als legaal aanschouwd kan worden onder de wetgeving van de GDPR. Onderzoek wees uit dat het systeem hier geen problemen van zou mogen ondervinden.

De resultaten van dit onderzoek wezen uit dat beide platformen hun eigen voor- en nadelen met zich meebrachten, verwijzend naar de verschillende functionaliteiten die onze zoekfunctie vereist. Een belangrijke factor bij het maken van de keuze is ook de eenvoud van de implementatie, waarbij al snel bleek dat het zeer moeilijk zal is om de sterktes van Neo4j te repliceren in Elasticsearch, of omgekeerd.

De resultaten zetten aan tot verder onderzoek naar de mogelijkheid om deze twee systemen parallel te gebruiken, en de resultaten ervan te combineren.  Zo kan Neo4j gebruikt worden als een 'Knowledge Graph', die dient als aanvulling voor de data die Elasticsearch tot zijn beschikking heeft. 




%---------- Inhoudstafel -------------------------------------------------------
\pagestyle{empty} % Geen hoofding
\tableofcontents  % Voeg de inhoudstafel toe
\cleardoublepage  % Zorg dat volgende hoofstuk op een oneven pagina begint
\pagestyle{fancy} % Zet hoofding opnieuw aan

%---------- Lijst figuren, afkortingen, ... ------------------------------------

% Indien gewenst kan je hier een lijst van figuren/tabellen opgeven. Geef in
% dat geval je figuren/tabellen altijd een korte beschrijving:
%
%  \caption[korte beschrijving]{uitgebreide beschrijving}
%
% De korte beschrijving wordt gebruikt voor deze lijst, de uitgebreide staat bij
% de figuur of tabel zelf.

\listoffigures
\listoftables

% Als je een lijst van afkortingen of termen wil toevoegen, dan hoort die
% hier thuis. Gebruik bijvoorbeeld de ``glossaries'' package.
% https://www.overleaf.com/learn/latex/Glossaries

%---------- Kern ---------------------------------------------------------------

% De eerste hoofdstukken van een bachelorproef zijn meestal een inleiding op
% het onderwerp, literatuurstudie en verantwoording methodologie.
% Aarzel niet om een meer beschrijvende titel aan deze hoofstukken te geven of
% om bijvoorbeeld de inleiding en/of stand van zaken over meerdere hoofdstukken
% te verspreiden!

%%=============================================================================
%% Inleiding
%%=============================================================================

\chapter{\IfLanguageName{dutch}{Inleiding}{Introduction}}
\label{ch:inleiding}

\section{\IfLanguageName{dutch}{Probleemstelling}{Problem Statement}}
\label{sec:probleemstelling}

Elke webshop of e-commerce toepassing die vandaag de dag gekend is maakt gebruik van een zoekfunctie die gebruikers toelaat om snel en eenvoudig producten te kunnen opzoeken. Deze zoekfunctie is beperkt tot de beschikbare informatie, en bijgevolg dus in de meeste gevallen gelimiteerd tot de productcatalogus.

 Rekeninghoudend met de algemene trend rond personalisering van de \textit{customer experience } zou het ook aangewezen zijn om ook de zoekfunctionaliteit op e-commerce websites te personaliseren. Dit zou resulteren in een betere customer experience voor de klant en een hogere conversie voor het bedrijf.
 
 Het personaliseren wordt in deze context bedoeld als het gebruiken van persoonlijke data om de resultaten van een zoekopdracht te verfijnen en zo producten te kunnen aanbieden die passen bij de voorkeuren en het gedrag van de gebruiker. 
 
 Hierop biedt dit onderzoek een mogelijke oplossing om dergelijke functionaliteiten te bereiken. Er zullen twee vooraf bepaalde platformen gebruikt worden om dit uit te werken, namelijk Neo4j en Elasticsearch. Het onderzoek zal uitwijzen welk platform het meest geschikt is om een gepersonaliseerde zoekfunctie mee te implementeren.
 
  Belangrijk hierbij is om na te gaan of de wetgeving omtrent GDPR het gebruik van de persoonlijke data als legaal aanschouwt in de context van zoekfuncties, meer specifiek op welke manier deze data gebruikt zal worden binnen de uitwerking van de twee voorgenoemde platformen.


\section{\IfLanguageName{dutch}{Onderzoeksvraag}{Research question}}
\label{sec:onderzoeksvraag}

De onderzoeksvraag bestaat eruit om te ontdekken welk van de twee vooraf gekozen platformen of welke technologie de beste oplossing biedt om in real-time op grote schaal gepersonaliseerde zoekresultaten te kunnen leveren.

Belangrijke criteria hierbij zijn snelheid, performantie, kwaliteit van de resultaten, en of het al dan niet mogelijk is om familierelaties te kunnen verwerken. Met performantie wordt hier geduid op de rekenkracht die een zoekoperatie zal nodig hebben, dit is gerelateerd aan de kost. 

Nog een niet te missen detail is de invloed van de wetgeving rond GDPR, en of een gepersonaliseerde zoekfunctie op basis van persoonlijke data al dan niet als legaal wordt aanzien onder deze wetgeving.

Concreet omvat dit onderzoek volgende vier onderzoeksvragen:
\begin{itemize}
	\item Is het implementeren van een gepersonaliseerde zoekfunctie op basis van persoonlijke data legaal onder de wetgeving rond GDPR?
	\item Welke technologie biedt de mogelijkheid om een gepersonaliseerde zoekfunctie te implementeren?
	\item Welke technologie biedt de beste resultaten op basis van snelheid, performantie en kwaliteit?
	\item Laten deze technologieën toe om rekening te houden met factoren die niet te maken hebben met historisch koopgedrag (bv. leeftijd, geslacht, gezinssamenstelling)?
\end{itemize} 

Deze vragen zullen doorheen dit onderzoek een antwoord krijgen.

\section{\IfLanguageName{dutch}{Onderzoeksdoelstelling}{Research objective}}
\label{sec:onderzoeksdoelstelling}

Het onderzoek heeft als doel om te ontdekken in hoeverre de persoonlijke data van een specifieke gebruiker en contextuele data kan ingeschakeld en gecombineerd worden met de data uit een productcatalogus om persoonlijke en relevante zoekresultaten te genereren.

Dit zal verwezenlijkt worden door middel een prototype op te stellen van beide vooraf gekozen platformen, zijnde Neo4j en Elasticsearch, en de mogelijkheden ervan te ontdekken. De ondervindingen van het onderzoek, naast de informatie die een literatuurstudie biedt, zullen een antwoord geven op de onderzoeksvragen.

Bij het gebruik van contextuele data wordt hier geduid op vooral historische koopdata van de persoon zelf, maar ook demografische data zoals geslacht, leeftijd, gezinssamenstelling kunnen hierbij belangrijke factoren zijn. 

Het is niet enkel de mogelijkheid om dergelijk zoeksysteem te verwezenlijken die van belang is voor dit onderzoek, ook het gebruiksgemak en eenvoud van installatie speelt hierbij een rol, en zullen ondanks een eerder subjectieve voorwaarde te zijn toch effect hebben op de conclusie.

\section{\IfLanguageName{dutch}{Opzet van deze bachelorproef}{Structure of this bachelor thesis}}
\label{sec:opzet-bachelorproef}

% Het is gebruikelijk aan het einde van de inleiding een overzicht te
% geven van de opbouw van de rest van de tekst. Deze sectie bevat al een aanzet
% die je kan aanvullen/aanpassen in functie van je eigen tekst.

De rest van deze bachelorproef is als volgt opgebouwd:

In Hoofdstuk~\ref{ch:stand-van-zaken} wordt een overzicht gegeven van de stand van zaken binnen het onderzoeksdomein, op basis van een literatuurstudie.

In Hoofdstuk~\ref{ch:methodologie} wordt de methodologie toegelicht en worden de gebruikte onderzoekstechnieken besproken om een antwoord te kunnen formuleren op de onderzoeksvragen.

% TODO: Vul hier aan voor je eigen hoofstukken, één of twee zinnen per hoofdstuk

In Hoofdstuk~\ref{ch:conclusie}, tenslotte, wordt de conclusie gegeven en een antwoord geformuleerd op de onderzoeksvragen. Daarbij wordt ook een aanzet gegeven voor toekomstig onderzoek binnen dit domein.
\chapter{\IfLanguageName{dutch}{Stand van zaken}{State of the art}}
\label{ch:stand-van-zaken}

% Tip: Begin elk hoofdstuk met een paragraaf inleiding die beschrijft hoe
% dit hoofdstuk past binnen het geheel van de bachelorproef. Geef in het
% bijzonder aan wat de link is met het vorige en volgende hoofdstuk.

% Pas na deze inleidende paragraaf komt de eerste sectiehoofding.

\section{Personalisatie}
\label{sec:Personalisatie}

Personalisatie van webapplicaties en websites draait erom de bezoekers een op maat gemaakte ervaring aan te bieden. Dit kan op verschillende manieren toegepast worden, en kan meerdere doelen hebben. In dit hoofdstuk wordt er verder ingegaan op welke manieren personalisatie van websites wordt toegepast, en wat dit betekent voor zowel de bezoeker als het bedrijf zelf.

\subsection{Personalisatie op basis van e-mail en sociale media}
\label{subsec:Personalisatie op basis van e-mail en Social Media}

Zowat iedereen heeft wel te kampen met een overvloed aan e-mails in hun postvak van allerlei websites waar ze hun e-mailadres ooit hebben vrijgegeven. U zou denken dat deze door de meeste mensen simpelweg verwijderd worden, maar e-mailmarketing blijft een van de meest succesvolle marketingstrategieën \autocite{Dehkordi2012}. E-mailmarketing is relatief makkelijk te implementeren en vereist weinig technische investering, meestal wordt dit verwezenlijkt via systemen van derden zoals bijvoorbeeld \href{https://mailchimp.com/}{MailChimp}. Een nadeel van deze marketingvorm is dat het bedrijf continu bezig moet zijn met nieuwe inhoud te creëren voor deze e-mails, alsook op de website waar de marketingmails over gaan. 

\subsection{Personalisatie op basis van geografische locatie}
\label{subsec:Personalisatie op basis van geografische locatie}

Geografische personalisatie is het aanpassen van de website op basis van de locatie van de gebruiker. Gebruikers uit België die naar de website van een internationaal bedrijf surfen, zullen dan worden omgeleid naar een Nederlandse of Franse versie van die website.

Geografische personalisatie kan ook gebruikt worden om de inhoud van een pagina aan te passen aan de hand van de locatie van de gebruiker, of om vertalingen aan te bieden.
Een nadeel hiervan is dat mensen die op reis gaan het soms moeilijk zouden kunnen hebben om naar de juiste versie van de website te navigeren, aangezien het systeem de gebruiker zal willen omleiden naar de pagina of inhoud die voorzien is voor het land waar zij zich momenteel in bevinden. Eenzelfde probleem kan zich voordoen bij bedrijven die hun webverkeer omleiden via een ander land door middel van bijvoorbeeld een VPN. 

Geografische personalisatie is ook relatief eenvoudig te implementeren en kan een grote troef zijn op de internationale markt. 

\subsection{Personalisatie op basis van IP-adres}
\label{subsec:Personalisatie op basis van IP-adres}

Deze methode van personalisatie is wat minder opvallend, aangezien het bij de gemiddelde internetgebruiker weinig tot nooit zal voorkomen, aangezien zij het internet gebruiken via een serviceprovider zoals Telenet of Proximus. 

Deze vorm van personalisatie wordt gebruikt om zakelijke gebruikers en bedrijven te kunnen identificeren op basis van hun IP-adres. Zo kan men zien of een bezoeker bij een bepaald bedrijf werkzaam is om deze direct aan te spreken op bijvoorbeeld de homepagina.

	\includegraphics[width=\linewidth]{img/e2fcfe784532c41a644e4465f535530d}

Net zoals bij personalisatie op basis van locatie kan dit misleidende resultaten opleveren, bijvoorbeeld als de werknemer van thuis werkt of het IP-adres niet duidelijk aantoont vanuit welk bedrijf het webverkeer van de bezoeker afkomstig is. Ook voor performantie kan dit negatieve gevolgen hebben, aangezien deze vorm van personalisatie afhankelijk is van systemen van derden. 

Verder moet er ook inhoud gecreëerd worden voor elk bedrijf dat men specifiek wil aanspreken. Dit is een tijdrovend proces, maar aangezien deze vorm van personalisatie weinig voorkomt, is het wel een troef waardoor het bedrijf zich kan onderscheiden van de meerderheid en zich kan laten opvallen.

  
 \subsection{Verwante inhoud personalisatie}
 \label{subsec:Verwante inhoud personalisatie}
 
 Dit is de vorm van personalisatie die een grote meerwaarde zal leveren aan dit onderzoek. De meeste mensen hebben deze vorm al ondervonden op een webshop zoals Amazon of Bol.com. Deze vorm draait erom de gebruikers artikels aan te raden op basis van artikels of inhoud die ze al eerder bekeken hebben, alsook het gedrag van andere gebruikers.
 
De werking van het aanbevelingssysteem van Amazon is gebaseerd op enkele complexe algoritmen  \autocite{Linden2003}. Dit is natuurlijk verantwoord omdat zij op zeer grote schaal werken en veel geld hebben geïnvesteerd in de ontwikkeling van hun systeem. 

In de realiteit hoeven de technologieën voor aanbevelingen van producten niet zo complex te zijn voor gewone webshops en bedrijven, vaak is het voldoende om relaties te creëren tussen artikels en op basis van deze relaties nieuwe artikels aan te raden aan de gebruikers. 

Een voorbeeld van een relatie tussen twee artikels is de welbekende 'Anderen bekeken ook' blok die vaak zichtbaar is bij het bekijken van een detailpagina van een product. 
Een simpelere methode van dergelijke relaties is het aanbieden van verwante producten op basis van categorieën of tags. Tags zijn een manier om kenmerken van een product weer te geven die specifieker zijn dan een categorie. Een categorie kan dan 'schoenen' zijn, terwijl een tag 'lage sneakers' is. 

\section{Recommender Systems}
\label{sec:Recommender Systems}
Recommender Systems \autocite{Resnick1997} zijn aanbevelingen vanuit het systeem die rekening houden met de beschikbare informatie van gebruikers en hun voorkeuren om zo een filter te plaatsen op de informatie die weergegeven wordt. Verder zullen we deze benoemen aan de hand  van hun Nederlandse naam 'aanbevelingssystemen'.

Aanbevelingssystemen worden vooral gebruikt in een e-commerce toepassingen waar een zeer groot en verscheiden aanbod aan producten is, en het al vaak lastig wordt om precieze aanbevelingen aan de klant te geven. Hierbij wordt allerlei informatie van een gebruiker verzameld, zoals historische aankopen, items op het verlanglijstje, items waar de gebruiker op geklikt heeft, etc.

In dit onderdeel zullen we kort wat dieper ingaan op de werking van dergelijke aanbevelingssystemen en de achterliggende algoritmen.

\subsection{Haalbaarheid}
\label{sec:Haalbaarheid}
Een degelijk systeem biedt een grote meerwaarde voor een bedrijf, zo heeft Netflix  een competitie gehouden die 1 miljoen dollar bood aan degene die een aanbevelingssysteem kon maken dat 10\% beter presteerde dan hun bestaande systeem. Deze wedstrijd liep van oktober 2006 tot minstens oktober 2011. De wedstrijd werd de \cite{NetflixPrize} genoemd, Netflix stelde hiervoor een dataset beschikbaar. Enkele groepen hebben het doel behaald, maar het algoritme van de winnende groep is uiteindelijk nooit in productie gebracht, omdat de mogelijke opbrengst niet kon opwegen tegen de extra gevraagde rekenkracht. 

Een aanbevelingssysteem moet dus niet enkel de juiste waarden kunnen aangeven, het moet ook haalbaar zijn qua rekenkracht en extra kosten. Het ontwerpen van dergelijk systeem is dus niet eenvoudig.

\subsection{Algoritmen voor aanbevelingssystemen}
\label{sec:Algoritmen voor aanbevelingssystemen}

We spreken over twee grote categorieën in algoritmen van aanbevelingssystemen: 'collaborative filtering'-algoritmen en 'content based'-algoritmen.  \cite{Adamovicius2005} Door de evolutie en groeiende ontwikkeling van beter presterende systemen zijn er ondertussen ook enkele algoritmen die niet echt binnen een van deze twee koepels vallen. In de onderstaande figuur wordt een simpele representatie gegeven van de werking van deze soorten algoritmes.

	\includegraphics[width=\linewidth]{img/Content-based-filtering-and-Collaborative-filtering-recommendation}
	
\subsubsection{Collaborative Filtering}
\label{sec:Collaborative Filtering}

Het kernidee bij Collaborative Filtering \autocite{Schafera} is om aanbevelingen te maken gebaseerd op de voorkeuren van een andere gebruiker met een gelijkaardig gedrag. Zoals de figuur hierboven aantoont dat als gebruiker 1 en 2 hetzelfde artikel gelezen hebben, gebruiker 2 een artikel als aanbeveling zal krijgen dat gelezen werd door gebruiker 1. 
Hetzelfde idee kan toegepast worden op allerlei interacties van de gebruikers met een website zoals likes, shares, verlanglijstjes, etc.

In de praktijk geeft deze techniek zeer goede resultaten, maar zoals verwacht brengt deze techniek ook enkele problemen met zich mee. Het meest merkwaardige probleem is een zogenaamde cold start, dit komt onder andere voor bij de eerste interactie van een nieuwe gebruiker met een applicatie die aanbevelingen biedt. Het algoritme heeft dan onvoldoende informatie om een correcte en nuttige aanbeveling op te leveren aan de gebruiker. Een andere oorzaak kan zijn dat er een nieuw product wordt toegevoegd aan het systeem, er kan dan nog niet geweten zijn welk type gebruiker hierin geïnteresseerd zou kunnen zijn. 

Een andere factor voor het succes van een Collaborative Filtering algoritme is het aantal gebruikers van een systeem, met andere woorden, hoe meer gebruikers er zijn, hoe correcter de aanbevelingen aan een specifieke gebruiker zal zijn. \autocite{Sarwar2001}

...

\subsubsection{Content Based}
\label{sec:Content Based}

TODO

\section{Wat is Graph?}
\label{sec:wat is Graph?}

In de context van deze bachelorproef zal er met 'Graph' steeds verwezen worden naar een Graph databank. In dit onderdeel zullen we wat dieper ingaan op wat dit soort databank precies inhoudt om een volledig begrijpen van de precieze werking te verzekeren. 

\subsection{Graph}
\label{sec:Graph}

Dit is de structuur die verder in dit onderzoek gebruikt zal worden. Deze structuur maakt gebruik van een wiskundige graaf om data op te slaan. Een graaf bestaat uit een aantal knopen (genaamd nodes) die al dan niet verbonden zijn. Een groot voordeel hiervan is dat deze databanken sneller zijn dan relationele databanken omdat ze snel naar een bepaalde node kunnen verwijzen, waarbij we bij een relationele databank een JOIN zouden moeten gebruiken. 

	\includegraphics[width=\linewidth]{img/Customer-Order-Example-Graph.png}
	
\subsection{Neo4j}
\label{sec:Neo4j}

Er bestaan verschillende platformen voor SQL, alsook verschillende platformen voor de hierboven beschreven NoSQL databanken. Neo4j is een van de meer bekende platformen voor graph databanken, dit platform zal gebruikt worden in dit onderzoek.

Walmart maakt gebruik van Neo4j om de aanbevelingen voor hun klanten op hun online webservices te optimaliseren \autocite{neo4jWalmart2014}. Zij gebruiken dit omdat graph databanken zeer snel over een gebruiker zijn koophistorie kunnen traverseren, en ook direct nieuwe mogelijke interesses kunnen halen uit het gedrag van de gebruiker. Daarmee wordt bedoeld dat er in real-time nieuwe connecties worden gelegd tussen de gebruiker en de producten, en hij de nieuwe aanbevelingen meteen zal zien, en niet enkele dagen of uren later. Er wordt dus historische data gematcht met real-time data, hier blinkt Neo4j in uit. 


Neo4j maakt dus gebruik van wiskundige grafen om data weer te geven samen met hun onderlinge relaties. Een graaf kan gericht of ongericht zijn, ongericht wil zeggen dat er geen richting is waarin de relaties lopen, dus deze zijn onderling uitwisselbaar. Een gerichte graaf is dan een graaf waarbij de relaties in een specifieke richting lopen, zoals volgers op Twitter: Persoon A volgt persoon B, maar persoon B volgt niet persoon A.


% TODO: Info van hieruit gebruiken en bronnen vermelden (zie pdf's van use cases)
% https://bbvaopen4u.com/en/actualidad/neo4j-what-graph-database-and-what-it-used


\section{Wat is ElasticSearch?}
\label{sec:wat is ElasticSearch?}

Elasticsearch op zichzelf is eigenlijk een zoekmachine die data kan analyseren, of dit nu nummers, tekst, gestructureerd of ongestructureerd is. Elasticsearch is een component van de Elastic stack, ook wel ELK-Stack genoemd. Dit staat voor Elasticsearch, Logstash en Kibana. Logstash wordt gebruikt om data te verwerken uit meerdere bronnen en te versturen naar Elasticsearch. Kibana is een tool om de gebruikers een visueel beeld te geven van de data in de vorm van grafieken of tabellen.

In Elasticsearch is het mogelijk gewichten toe te kennen aan de resultaten van een zoekopdracht, zo kunnen meer relevante resultaten hoger in de lijst staan, dit is een belangrijke factor in dit onderzoek, namelijk of deze technologie gebruikt kan worden in E-Commerce toepassingen. In het artikel van \cite{Vavliakis2019} wordt beweerd dat het systeem dat zij implementeerden op een performante manier de gewenste resultaten gaf, alsook dat dit een oplossing kan zijn voor real-time zoekopdrachten in commerciële toepassingen.

...
%%=============================================================================
%% Methodologie
%%=============================================================================


\chapter{\IfLanguageName{dutch}{Methodologie}{Methodology}}
\label{ch:methodologie}


Rekeninghoudend met de bevindingen uit de literatuurstudie, zal hier verder besproken worden hoe we een systeem gaan opstellen dat gepaste aanbevelingen kan maken. Gezien we twee verschillende werkwijzen en technologieën zullen vergelijken, zal voor beide systemen een werkwijze voorgesteld worden. Deze werkwijze zal ook in dit hoofdstuk verder uitgewerkt worden.

\section{Uitwerking ElasticSearch}
\label{sec:Uitwerking ElasticSearch}

Voor de uitwerking van Elasticsearch wordt een instantie opgezet waarmee we zullen communiceren, op deze instantie draait Elasticsearch. 

In Elasticsearch worden items opgeslagen als een document met enkele waarden, typisch aan een product. In dit onderzoek zijn deze waarden; 'naam', 'merk', 'categorieën', en 'prijs'. 

\subsection{Producten}
\label{sec:Producten}
Voor het aanmaken van de product data zullen enkele query's uitgevoerd worden die er als volgt uitzien:

\newpage
\begin{lstlisting}[caption={Query om één enkel product aan te maken},captionpos=b]
POST http://35.233.112.106:9200/products/product/1 
{ 
 "name": "Logitech G930",
 "brand": "Logitech",
 "categories": ["Headset", "Wireless headset", "Headphones"],
 "price": "190.00"
}
\end{lstlisting}

Op deze manier worden enkele producten in de databank gezet, met enkele verwante velden zoals gelijkaardige categorieën, zodat we hiermee aanbevelingen kunnen maken. Deze verzameling van producten wordt een index genoemd.

Vervolgens kunnen we een zoekterm ingeven via volgende query, deze zal alle resultaten weergeven waarvan een van de velden voldoet aan de meegegeven waarde, namelijk "deodorant". 

\begin{lstlisting}[caption={Query om één enkel product op te halen},captionpos=b]
GET http://35.233.112.106:9200/products/product/1 
{
 "query" : {
  "query_string": {
   "query": "deodorant"
  }
 }
}
\end{lstlisting}

\subsection{Gebruikers}

Er zullen enkele vooraf gedefinieerde gebruikers opgesteld worden, waarmee bepaald zal worden of de zoekresultaten aan de verwachtingen voldoen. Een voorbeeld van een gebruiker is te zien in volgende query:
\begin{lstlisting}[caption={Query om één enkele gebruiker aan te maken}]
POST http://35.233.112.106:9200/users/user/1
{
	"name": "Louise",
	"age": "21",
	"address": {
		"city": "Aalst",
		"street": "Molendries 4",
		"province": "Oost-Vlaanderen"
	},
	"categories": ["Electronics, Apple"],
	"searches": ["iphone", "deodorant", "uncommon product"],
	"brands": ["Nivea", "Apple"]
}
\end{lstlisting}

Bij deze gebruiker kunnen we bijvoorbeeld verwachten dat als deze 'deodorant' opzoekt, zij die van het merk Nivea bovenaan de resultaten zal zien. 

Een voorbeeld van zo'n zoekopdracht, waarbij sommige velden ingevuld zijn op basis van onze persoon van de query hierboven, gaat als volgt:

\begin{lstlisting}[caption={Query om een zoekopdracht met term 'Deodorant' uit te voeren, met filters en een score op basis van informatie van een gebruiker}]
POST http://35.233.112.106:9200/products/_search
{
  "query": {
    "function_score": {
      "query": {
        "query_string": {
          "query": "Deodorant"
        }
      },
      "functions": [
        { "filter" : 
          { "terms" : 
            { "brand" : ["Niveau", "Apple"]  } 
          },
          "weight": 3
        },
        { "filter" : 
          { "terms" : 
            { "categories" : ["Electronics", "Apple"] }
          },
          "weight": 2
        },
        { "filter" : 
          { "terms" : 
            { "searches" : ["iphone", "deodorant", "uncommon product"] } 
          },
          "weight": 1
        }
      ],
      "score_mode": "sum",
      "boost_mode": "replace"
      }
   }
}
\end{lstlisting}

In het bovenste deel van de query, waar de \textit{query\textunderscore string} wordt aangeduid. Deze string zou dan overeenkomen met wat een gebruiker zou invoeren in een zoekbalk op een website.

In de realiteit zal de informatie over 'brand', 'categories', en 'searches' opgehaald worden uit het model van de persoon in kwestie.

In deze query wordt een score toegekend aan de resultaten die voldoen aan bovenstaande verwachting, namelijk dat het woord 'deodorant' terug te vinden moet zijn in een van de velden van het product (naam, merk, categorie)

\newpage
Die score wordt toegekend op basis van de informatie over een persoon, hieruit verstaan we dat dit gaat over de merken die deze persoon reeds gekocht heeft, in welke categorieën deze persoon reeds gekocht heeft, en een historiek van zoekopdrachten. Deze hebben elk een gewicht toegekend gekregen, respectievelijk drie, twee en één. 

In de query zijn drie filters te zien die de score van een resultaat zullen beïnvloeden. Bij elk van deze filters wordt het score vermenigvuldigd met het gewicht. \textit{score\textunderscore mode : sum} zorgt ervoor dat de resultaten van de scores opgeteld worden.

\textit{boost\textunderscore mode : replace} zorgt ervoor dat de score die verkregen wordt bij het gelijkaardig zijn aan de zoekterm vervangen wordt door de nieuw berekende score van de functies.

Deze query is een vorm van collaborative filtering, toegepast op zichzelf. De query zal dus gaan zoeken naar hoe verwant een resultaat is aan een persoon, op basis van enkele variabelen, en zal deze een hogere score toekennen afhankelijk van hoe relevant het product is voor een gebruiker. Resultaten met een hogere score zullen dus bijgevolg ook hoger in de lijst van resultaten komen te staan.

\subsection{Graph API Plug-in}
\label{sec:Graph API Plug-in}

Elasticsearch biedt ook een plug-in aan die de functionaliteiten van een graph database voorziet. Met deze plug-in bekomen we dus eigenlijk een Knowledge Graph zoals eerder in de literatuurstudie vermeld werd. 

Deze plug-in laat toe om connecties te vinden tussen objecten in de databank, en wordt geadverteerd als ook bruikbaar te zijn voor aanbevelingen.

Jammer genoeg is deze plug-in enkel beschikbaar wanneer we beschikken over een Platinum of Enterprise licentie. Door deze vereiste valt deze functionaliteit dus buiten de scope van het onderzoek. 

\begin{figure} [ht]
	\centering
	\includegraphics[width=0.95\textwidth]{img/elastic-license}
	\caption{Overzicht van de beschikbaarheid van de Graph plugin voor de Elastic licenties}
	\floatfoot{Source: https://www.elastic.co/subscriptions}
	\label{fig:elastic licenties overzicht graph}
\end{figure}

\newpage
\section{Uitwerking Neo4j}
\label{sec:UItwerking Neo4j}

Voor dit systeem wordt er een model opgesteld in een Graph databank, de gekozen technologie hiervoor is Neo4j. Op deze databank kunnen dan zoekalgoritmen uitgevoerd worden, om zo tot correcte aanbevelingen te komen. In dit hoofdstuk zal verder uitgewerkt worden hoe dit precies in elkaar zit.

\subsection{Model}
\label{sec:Model}
Relaties tussen producten en klanten zullen als volgt worden opgeslagen in deze Graph databank:

Er zijn 2 soorten nodes, dit zijn de knopen van een graaf, namelijk 'Gebruiker' en 'Product'. De relaties, aangeduid door lijnen tussen de knopen, zullen dan voorstellen wat de interactie van een gebruiker met een product is. Dit kan 'LIKES', 'BOUGHT', of 'LIVES\_TOGETHER' zijn. 'LIVES\_TOGETHER' zal dan aanduiden of 2 gebruikers familie of samenwonend zijn. 

De attributen van een Product zijn als volgt:
\begin{itemize}
	\item Product ID
	\item Name
	\item Brand
\end{itemize}

De attributen voor een Gebruiker zijn als volgt:
\begin{itemize}
	\item Name
	\item Address
\end{itemize}

De verschillende soorten relaties zijn als volgt:
\begin{itemize}
	\item 'BOUGHT' -> Gebruiker heeft Product gekocht
	\item 'LIKES' -> Gebruiker heeft Product leuk gevonden
	\item 'LIVES\_TOGETHER' -> Gebruiker 1 heeft hetzelfde adres als Gebruiker 2
\end{itemize}

Volgende query wordt gebruikt om de product- en klant nodes aan te maken, dit is uiteraard een verkorte versie:

\begin{lstlisting}[caption={Neo4j query voor het aanmaken van producten en klanten}]
CREATE 
(shauni:Customer {name: 'Shauni', address: 'Exterkenstraat 14'}),
(lynn:Customer {name: 'Lynn', address: 'Exterkenstraat 14'}),
(angelo:Customer {name: 'Angelo', address: 'Arbeidstraat 14'}), 
...
(prod1:Product{id: '1', name:'Hairbrush', brand:'Syoss'}), 
(prod2:Product{id: '2', name:'Instant Chocolate Milk', brand:'Nesquick'}), 
(prod3:Product{id: '3', name:'Toilet Paper', brand: 'Boni'}), 
...
\end{lstlisting}

Om een relatie aan te maken tussen een product en een gebruiker zal volgende query gebruikt worden. Deze zal dus uitgevoerd worden elke keer een gebruiker een actie uitvoert met een product.
Aangezien een product leuk vinden en een product kopen niet aanzien worden als evenwaardig, zullen er gewichten toegekend worden aan de interacties tussen gebruikers en producten.  Dat kan door middel van volgende query:

\begin{lstlisting}[caption={Neo4j query voor het aanmaken van een relatie tussen een product en een klant}]
	MATCH (c:Customer),(p:Product) 
	WHERE c.name = 'Shauni' AND p.id=3 
	CREATE (c)-[r:BOUGHT]->(p) 
	SET r.score = 3
\end{lstlisting}

Een product leuk vinden krijgt een score 2 toegekend, en een product kopen zal de score 3 krijgen, om de relatie 'LIKES' aan te maken zal in bovenstaande query de relatie 'BOUGHT' en de score moeten aangepast worden.

Om de relatie van Gebruikers die op eenzelfde adres wonen aan te maken, gebruiken we volgende query: 

\begin{lstlisting}[caption={Neo4j query voor het aanmaken van een relatie tussen een twee klanten.}]
	MATCH (a:Customer), (b:Customer) 
	WHERE EXISTS (a.address) AND EXISTS (b.address) AND a.address=b.address AND id(a)<id(b) 
	CREATE (a)-[:LIVES\_TOGETHER]->(b); 
\end{lstlisting}

De uiteindelijke graaf die voor dit zeer klein voorbeeld opgebouwd werd, zal al snel een beter inzicht geven in hoe deze data aan elkaar vast hangt. Dit is ook een van de troeven van Neo4j, het voorstellen van duidelijke visuele beelden die het eenvoudig maken om te begrijpen wat er zich op de achtergrond afspeelt. De graaf is te zien in Figuur \ref{fig:neo4jfullgraph}
\newpage

\begin{figure} [h!]
	\centering
	\includegraphics[width=0.95\textwidth]{img/full_graph_result}
	\caption{Overzicht van de ingevoerde data in een Neo4j graaf}
	\label{fig:neo4jfullgraph}
\end{figure}


\subsection{Ophalen van data}
\label{sec:Neo4j Ophalen van data}

Voor deze uitwerking werden enkele query's opgesteld die op basis van de relaties producten retourneren die zouden kunnen dienen als aanbevelingen.

Zo kan er bijvoorbeeld een lijst van producten gegenereerd worden die leuk gevonden of gekocht werden door de familieleden van een bepaalde gebruiker, dit gebeurt op basis van de relatie LIVES\_TOGETHER.

\begin{lstlisting}[caption={Neo4j query die alle likes en aankopen van de familie van een bepaalde gebruiker c1 weergeeft}]
MATCH (c1: Customer)
WHERE c1.name = 'Shauni'
MATCH (c1) - [:LIVES_TOGETHER] -> (c2 : Customer) - [:BOUGHT] -> (p:Product)
RETURN p as Product
UNION
MATCH (c1) - [:LIVES_TOGETHER] -> (c2) - [:LIKES] -> (p:Product)
RETURN p as Product
\end{lstlisting}

Ook van belang is het ophalen van producten die gebruikers kopen waar een gebruiker \textit{c1} gemeenschappelijke aankopen mee heeft. Dit kan via volgende query:

\begin{lstlisting}[caption={Neo4j query die de aankopen van gebruikers weergeeft waar een gebruiker c1 gemeenschappelijke aankopen mee heeft}]
MATCH (c1: Customer)
WHERE c1.name = 'Shauni'
MATCH(c1) - [:BOUGHT] -> (p1 : Product) <- [:BOUGHT] - (c2: Customer) - [:BOUGHT] -> (p2: Product)
RETURN p2
\end{lstlisting}

Om ditzelfde te bereiken maar dan inclusief de likes van deze gebruikers waar een gebruiker \textit{c1} gemeenschappelijke aankopen mee heeft, kan volgende query gebruikt worden: 

\begin{lstlisting}[caption={Neo4j query die de aankopen en likes van gebruikers weergeeft waar een gebruiker c1 gemeenschappelijke aankopen mee heeft}]
MATCH (c1) - [:BOUGHT] -> (p1 : Product) <- [:BOUGHT] - (c2: Customer) - [:BOUGHT] -> (p2: Product)
RETURN p2 as Product
UNION ALL
MATCH (c1) - [:BOUGHT] -> (p1) <- [:BOUGHT] -(c2) - [:LIKES] -> (p2) 
WHERE id(c1)<id(c2)
RETURN p2 as Product
\end{lstlisting}


\subsection{Search}
\label{subsec: Search Neo4j}

Binnen neo4j is het ook mogelijk om zoekopdrachten uit te voeren op basis van een tekstveld. Hiervoor moet op voorhand een index aangemaakt worden waarin beschreven staat welke attributen van welke types van nodes gebruikt mogen worden om de ingevoerde tekst mee te vergelijken.

\begin{lstlisting}[caption={Een index creëren binnen Neo4j om zoekopdrachten op basis van tekst uit te voeren}]
CALL db.index.fulltext.createNodeIndex("namesAndBrands",["Product"],["name", "brand"])
\end{lstlisting}

In dit voorbeeld gebruiken we de attributen \textit{name} en \textit{brand} van het nodetype \textit{Product}, en nemen we als zoekterm ``deodorant``.  Op deze manier kan er dus eenvoudig gekozen worden op welke velden van de objecten er moet vergeleken worden.



\begin{lstlisting}[caption={Een zoekopdracht uitvoeren op basis van tekst}]
CALL db.index.fulltext.queryNodes("namesAndCategories", "deodorant") YIELD node, score
RETURN node.name, node.category, score
\end{lstlisting}

Er zal een lijst geretourneerd worden met de attributen (node.name), en een score die door het algoritme werd toegekend die aantoont hoe sterk het resultaat gelijkend is aan de ingevoerde tekst.

\subsection{PageRank}
\label{subsec:PageRank}

Voor het vinden van resultaten die belang hebben voor de gebruiker, wordt het PageRank algoritme gebruikt. Dit algoritme berekent de belangrijkheid van een knoop op basis van zijn buren. Hoe meer connecties er naartoe, hoe sterker het resultaat. Op deze manier worden populaire resultaten hoger gerangschikt.

Om via dit algoritme aanbevelingen te verkrijgen, moet er eerst een nieuwe graaf worden opgesteld waar enkel de informatie in zit die nodig is voor de berekening, dit gebeurt via volgende query, waarin de interacties 'BOUGHT' en 'LIKES' opgenomen worden, samen met hun gewicht (score).

\begin{lstlisting}[caption={Een genoemde graaf creëren om graafalgoritmen op uit te voeren}]
CALL gds.graph.create.cypher(
	'boughtGraph',
	'MATCH (p:Product) RETURN id(p) AS id',
	'MATCH (p2:Product)<-[r:BOUGHT]-(c:Customer)-[:BOUGHT|LIKES]->(p3:Product)
	RETURN
		id(p2) AS source,
		id(p3) AS target,
		r.score AS score',
		{
			readConcurrency: 4
		}
)
\end{lstlisting}

Vervolgens kunnen we het PageRank algoritme uitvoeren op deze graaf, die een lijst van productnamen zal weergeven met daarnaast een score.

\begin{lstlisting}[caption={PageRank algoritme uitvoeren}]
CALL gds.pageRank.stream('boughtGraph', { maxIterations: 3, dampingFactor: 0.85 })
YIELD nodeId, score
RETURN gds.util.asNode(nodeId).name AS name, score
ORDER BY score DESC, name ASC
\end{lstlisting}

Het resultaat van dit algoritme (Figuur \ref{fig:simplePageRankResult}) toont ons dat ``Wireless Mouse`` de belangrijkste knoop is in deze graaf, met andere woorden dat als er door de graaf gelopen wordt, deze knoop de grootste kans heeft om bereikt te worden. 

\begin{figure} [ht]
	\centering
	\includegraphics[width=0.95\textwidth]{img/pageRank_res_1}
	\caption{Resultaat PageRank algoritme}
	\label{fig:simplePageRankResult}
\end{figure}

\newpage
\subsection{Personalized PageRank}
\label{sec:Personalized PageRank}

Om de resultaten te baseren vanuit een bepaald startpunt, bijvoorbeeld een Customer, kan er een variant op PageRank gebruikt worden. Het algoritme zal dan starten in een bepaalde knoop (de gebruiker in kwestie) en zal op deze manier de meest relevante knopen vinden.

We gebruiken opnieuw de genoemde graaf uit de simpele PageRank, en voeren daar volgend algoritme op uit: 

\begin{lstlisting}[caption={Personalized PageRank algoritme }]
MATCH (c1:Customer {name: 'Tony'})
CALL gds.pageRank.stream('boughtGraph', {
	maxIterations: 1,
	dampingFactor: 0.85,
	sourceNodes: [c1]
	})
YIELD nodeId, score
RETURN gds.util.asNode(nodeId).name AS name, score
ORDER BY score DESC, name ASC
\end{lstlisting}

In het resultaat van deze query (Figuur \ref{fig:personalizedPageRankResult}) zien we dat de resultaten voor deze gebruiker in dezelfde lijn lopen als de algemene resultaten, maar dat er toch enkele producten bovenaan de lijst te vinden zijn die bij het gewone PageRank algoritme niet aan bod komen.

\begin{figure} [ht]
	\centering
	\includegraphics[width=0.95\textwidth]{img/persPageRank_res_1}
	\caption{Resultaat Personalized PageRank algoritme}
	\label{fig:personalizedPageRankResult}
\end{figure}

\newpage
\subsection{Community Detection}
\label{subsec: Community Detection}

Door het analyseren van de graaf met een community detection-algoritme kunnen groepen van gelijkaardige producten ontdekt worden. Op deze manier kan ontdekt worden welke gebruikers geïnteresseerd zijn in bepaalde groepen van producten, en kunnen er op basis van deze clusters aanbevelingen voorgesteld worden. 

Een nadeel aan deze techniek is dat deze last heeft van het cold-start probleem. Met andere woorden dat voor nieuwe gebruikers of producten, waarvoor dus nog niet bekend is tot welke cluster ze behoren of in welke clusters zij interesse hebben, nog geen correcte aanbevelingen kunnen bepaald worden.

Een voorbeeld van een community detection algoritme is het Louvain algoritme. De genoemde graaf van de uitwerking van PageRank zal opnieuw worden gebruikt voor dit algoritme.

\begin{lstlisting}[caption={ Louvain algoritme }]
CALL gds.louvain.stream('boughtGraph')
YIELD nodeId, communityId, intermediateCommunityIds
RETURN gds.util.asNode(nodeId).name AS name, communityId, intermediateCommunityIds
ORDER BY communityId DESC
\end{lstlisting}

Uit de resultaten worden enkele clusters ontdekt, deze zijn aangeduid op Figuur \ref{fig:LouvainResult}.

\begin{figure} [ht]
	\centering
	\includegraphics[width=0.95\textwidth]{img/Louvain_result}
	\caption{Resultaat Louvain algoritme}
	\label{fig:LouvainResult}
\end{figure}





% Voeg hier je eigen hoofdstukken toe die de ``corpus'' van je bachelorproef
% vormen. De structuur en titels hangen af van je eigen onderzoek. Je kan bv.
% elke fase in je onderzoek in een apart hoofdstuk bespreken.

%\input{...}
%\input{...}
%...

%%=============================================================================
%% Conclusie
%%=============================================================================

\chapter{Conclusie}
\label{ch:conclusie}

% TODO: Trek een duidelijke conclusie, in de vorm van een antwoord op de
% onderzoeksvra(a)g(en). Wat was jouw bijdrage aan het onderzoeksdomein en
% hoe biedt dit meerwaarde aan het vakgebied/doelgroep? 
% Reflecteer kritisch over het resultaat. In Engelse teksten wordt deze sectie
% ``Discussion'' genoemd. Had je deze uitkomst verwacht? Zijn er zaken die nog
% niet duidelijk zijn?
% Heeft het onderzoek geleid tot nieuwe vragen die uitnodigen tot verder 
%onderzoek?

\section{ ElasticSearch}
\label{sec:Conclusie Elasticsearch}

Elasticsearch is zeer snel in het vinden van producten die relateren aan de gebruiker zelf, maar is wat moeilijker in omgang met het gebruik maken van data die personen aan elkaar linkt. Binnen dit onderzoek is er geen manier ontdekt waarbij er rekening kan gehouden worden met familiale verbanden. 


\section{ Neo4j}
\label{sec:Conclusie Neo4j}

%%=============================================================================
%% Bijlagen
%%=============================================================================

\appendix
\renewcommand{\chaptername}{Appendix}

%%---------- Onderzoeksvoorstel -----------------------------------------------

\chapter{Onderzoeksvoorstel}

Het onderwerp van deze bachelorproef is gebaseerd op een onderzoeksvoorstel dat vooraf werd beoordeeld door de promotor. Dat voorstel is opgenomen in deze bijlage.

% Verwijzing naar het bestand met de inhoud van het onderzoeksvoorstel
%---------- Inleiding ---------------------------------------------------------

\section{Introductie} % The \section*{} command stops section numbering
\label{sec:introductie}

De meeste e-commerce online platformen hebben reeds een zoekfunctie ingebouwd, maar deze is vaak beperkt tot de  enkel de productcatalogus. Rekening houdend met de algemene trend rond personalisering van de customer experience zou het ook aangewezen zijn om ook de search op de e-commerce site te personaliseren. Dit zou resulteren in een betere experience voor de klant en een hogere conversie voor het bedrijf. In deze bachelorproef wordt onderzocht in hoeverre de persoonlijke data van een specifieke gebruiker en contextuele data kan ingeschakeld en gecombineerd worden met de data uit de productcatalogus om persoonlijke en relevante zoekresultaten te genereren. 
Met deze persoonlijke data wordt bedoeld de historische aankoopdata van de persoon zelf, maar ook bepaalde demografische data zoals geslacht, leeftijd, gezinssamenstelling, etc. 


%---------- Stand van zaken ---------------------------------------------------

\section{State-of-the-art}
\label{sec:state-of-the-art}

Personalised Search \autocite{Pitkow2002} verwijst naar het zoeken op het web waarbij de resultaten afhankelijk zijn van de interesses en voorkeuren van de gebruiker die verder gaan dan de query zelf. 

Personalisatie in e-commerce toepassingen biedt een groot voordeel aan bedrijven. De loyaliteit van klanten wordt veel sterker als deze gebruik maken van gepersonaliseerde features.\autocite{Telang2005} Een zoekfunctie is een voorbeeld van zo een feature.
Recommender Systems \autocite{Resnick1997} zijn aanbevelingen vanuit het systeem die rekening houden met de beschikbare informatie van gebruikers en hun voorkeuren om zo een filter te plaatsen op de informatie die weergegeven wordt. 

De studie van \textcite{Diehl2003} onderzocht het effect van gepersonaliseerde zoekresultaten op de kwaliteit van keuzes die klanten maken, en vond een positieve correlatie. De studie ontdekte dat het verlagen van search cost \autocite{Smith1999} leidde tot minder kwaliteitsvolle keuzes. De reden daarvoor is dat klanten slechtere beslissingen maken als de search cost lager ligt omdat zij minder ideale opties aangeboden krijgen. Personalised Search en Recommender System zorgen voor een enorme verbetering in de kwaliteit van de keuzes die de klant maakt, en verminderen het aantal producten die deze klant bekijkt alvorens hij/zij gevonden heeft wat hij/zij nodig heeft.

Een gevolg van gepersonaliseerde zoekopdrachten is dat we een Filter Bubble \autocite{Pariser2011} creëren. Het verlaagt de kans dat nieuwe informatie gevonden wordt doordat de resultaten van een zoekopdracht partijdig zijn en eerder wijzen naar dingen die de gebruiker reeds gezien heeft. Dit concept wordt een Filter Bubble genoemd omdat gebruikers eigenlijk geïsoleerd worden in hun eigen wereldje, waar ze enkel de informatie te zien krijgen die ze willen zien. Als we deze gebruikers met hun bubbels in groepen opdelen, verkrijgen we wel het probleem dat deze een vertekend beeld op de realiteit krijgen, zij krijgen bijvoorbeeld in het nieuws slechts het deel te zien dat voor hun interessant is. Als we deze lijn doortrekken naar de klanten van e-commerce websites, zullen zij ook slechts de merken te zien krijgen waar hun voorkeur naar uit gaat. Hierdoor verminder je de kans dat ze een nieuw merk ontdekken of een ander product uitproberen. Als we terug refereren naar de studie van \textcite{Diehl2003}, dan kunnen we afleiden dat dit een positief effect zal hebben op de klanttevredenheid.

% Voor literatuurverwijzingen zijn er twee belangrijke commando's:
% \autocite{KEY} => (Auteur, jaartal) Gebruik dit als de naam van de auteur
%   geen onderdeel is van de zin.
% \textcite{KEY} => Auteur (jaartal)  Gebruik dit als de auteursnaam wel een
%   functie heeft in de zin (bv. ``Uit onderzoek door Doll & Hill (1954) bleek
%   ...'')


%---------- Methodologie ------------------------------------------------------
\section{Methodologie}
\label{sec:methodologie}

Om de onderzoeksvraag te beantwoorden wordt er een simpele webapplicatie opgezet waar een zoekterm kan ingevoerd worden. Ook worden er twee databankmodellen ontworpen, een model dat gebruik maakt van Neo4j (Graph databank platform), en een model dat gebruik maakt van Elasticsearch (OLAP databank platform) en Kibana om de resultaten te visualiseren. Deze zullen elk met hun eigen API communiceren, en in beide modellen wordt dezelfde gebruiker- en productdata ingevoerd. Bij het opstellen van de modellen wordt mogelijk al duidelijk of één van de modellen niet in staat zal zijn om dezelfde functionaliteiten te hebben als het andere, en dan zal moeten afgewogen worden of de voordelen van het ene model opwegen tegenover het andere model. Als beide modellen dezelfde functionaliteit kunnen bereiken, wordt er bekeken welke het meest performante is. Mogelijks zou het ook haalbaar zijn om beide manieren te combineren op voorwaarde dat de responstijd binnen de acceptabele norm valt.


%---------- Verwachte resultaten ----------------------------------------------
\section{Verwachte resultaten}
\label{sec:verwachte_resultaten}

Er wordt verwacht dat er ofwel een duidelijk verschil merkbaar is in performantie tussen de twee verschillende modellen. Mogelijk is dat één van de twee modellen totaal niet haalbaar is om efficiënt een link mee te leggen tussen bijvoorbeeld familieleden, in dit geval bekijken we of het mogelijk is deze twee modellen samen uit te voeren, als dit een resultaat biedt dat binnen de norm valt qua performantie, dan is dit ook een mogelijke oplossing.
 Het kan zich ook voordoen dat beide modellen vrij performant en efficiënt de queries kunnen verwerken, in dit geval worden de voor- en nadelen alsook de moeilijkheidsgraad van implementatie afgewogen.


%---------- Verwachte conclusies ----------------------------------------------
\section{Verwachte conclusies}
\label{sec:verwachte_conclusies}

Er wordt verwacht dat het Graph model makkelijker en beter zal presteren als we rekening willen houden met het aankoopgedrag van vrienden en familie. Indien dit ook mogelijk is bij een OLAP-model, verwachten we dat Graph nog steeds beter zal presteren. Indien we hier geen rekening mee houden zal het OLAP-model beter presteren, aangezien dit model aangepast is aan grote hoeveelheden data waar complexe zoekopdrachten op kunnen uitgevoerd worden.




%%---------- Andere bijlagen --------------------------------------------------
% TODO: Voeg hier eventuele andere bijlagen toe
%\input{...}

%%---------- Referentielijst --------------------------------------------------

\printbibliography[heading=bibintoc]

\end{document}
