%%=============================================================================
%% Inleiding
%%=============================================================================

\chapter{\IfLanguageName{dutch}{Inleiding}{Introduction}}
\label{ch:inleiding}

Deze inleiding is onderverdeeld in enkele secties die duidelijkheid zullen verschaffen over de opzet van deze bachelorproef en het onderzoek. Deze secties zijn als volgt:

\begin{itemize}
  \item context, achtergrond
  \item afbakenen van het onderwerp
  \item verantwoording van het onderwerp, methodologie
  \item probleemstelling
  \item onderzoeksdoelstelling
  \item onderzoeksvraag
  \item \ldots
\end{itemize}

\section{\IfLanguageName{dutch}{Probleemstelling}{Problem Statement}}
\label{sec:probleemstelling}

De meeste e-commerce online platformen hebben reeds een zoekfunctie geïmplementeerd,  maar deze is vaak beperkt tot enkel de productcatalogus. Rekeninghoudend met de algemene trend rond personalisering van de customer experience zou het ook aangewezen zijn om ook de zoekfunctionaliteit op e-commerce websites te personaliseren. Dit zou resulteren in een betere  gebruikservaring voor de klant en een hogere conversie voor het bedrijf.

\section{\IfLanguageName{dutch}{Onderzoeksvraag}{Research question}}
\label{sec:onderzoeksvraag}

De onderzoeksvraag bestaat eruit om te ontdekken welke databanktechnologie de beste oplossing biedt om in real-time op grote schaal gepersonaliseerde zoekresultaten te kunnen leveren.

Belangrijke criteria hierbij zijn performantie, kwaliteit van de resultaten, en of het al dan niet mogelijk is om familierelaties te kunnen verwerken. 

Concreet omvat dit onderzoek volgende onderzoeksvragen:
\begin{itemize}
	\item Welke technologie biedt de mogelijkheid om een gepersonaliseerde zoekfunctie te implementeren
	\item Welke technologie biedt de beste resultaten op basis van performantie en kwaliteit
	\item Laten deze technologieën toe om rekening te houden met factoren die niet te maken hebben met historisch koopgedrag (bv. leeftijd, geslacht, gezinssamenstelling)

\end{itemize} 

\section{\IfLanguageName{dutch}{Onderzoeksdoelstelling}{Research objective}}
\label{sec:onderzoeksdoelstelling}

Het onderzoek heeft als doel om te ontdekken in hoeverre de persoonlijke data van een specifieke gebruiker en contextuele data kan ingeschakeld en gecombineerd worden met de data uit een productcatalogus om persoonlijke en relevante zoekresultaten te genereren.
We spreken hier vooral over historische koopdata van de persoon zelf, maar ook demografische data zoals geslacht, leeftijd, gezinssamenstelling kunnen hierbij belangrijke factoren zijn. 

\section{\IfLanguageName{dutch}{Opzet van deze bachelorproef}{Structure of this bachelor thesis}}
\label{sec:opzet-bachelorproef}

% Het is gebruikelijk aan het einde van de inleiding een overzicht te
% geven van de opbouw van de rest van de tekst. Deze sectie bevat al een aanzet
% die je kan aanvullen/aanpassen in functie van je eigen tekst.

De rest van deze bachelorproef is als volgt opgebouwd:

In Hoofdstuk~\ref{ch:stand-van-zaken} wordt een overzicht gegeven van de stand van zaken binnen het onderzoeksdomein, op basis van een literatuurstudie.

In Hoofdstuk~\ref{ch:methodologie} wordt de methodologie toegelicht en worden de gebruikte onderzoekstechnieken besproken om een antwoord te kunnen formuleren op de onderzoeksvragen.

% TODO: Vul hier aan voor je eigen hoofstukken, één of twee zinnen per hoofdstuk

In Hoofdstuk~\ref{ch:conclusie}, tenslotte, wordt de conclusie gegeven en een antwoord geformuleerd op de onderzoeksvragen. Daarbij wordt ook een aanzet gegeven voor toekomstig onderzoek binnen dit domein.