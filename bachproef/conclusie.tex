%%=============================================================================
%% Conclusie
%%=============================================================================

\chapter{Conclusie}
\label{ch:conclusie}

% TODO: Trek een duidelijke conclusie, in de vorm van een antwoord op de
% onderzoeksvra(a)g(en). Wat was jouw bijdrage aan het onderzoeksdomein en
% hoe biedt dit meerwaarde aan het vakgebied/doelgroep? 
% Reflecteer kritisch over het resultaat. In Engelse teksten wordt deze sectie
% ``Discussion'' genoemd. Had je deze uitkomst verwacht? Zijn er zaken die nog
% niet duidelijk zijn?
% Heeft het onderzoek geleid tot nieuwe vragen die uitnodigen tot verder 
%onderzoek?

\section{ ElasticSearch}
\label{sec:Conclusie Elasticsearch}

Elasticsearch is zeer snel in het vinden van producten die relateren aan de gebruiker zelf, maar is wat moeilijker in omgang met het gebruik maken van data die personen aan elkaar linkt. Binnen dit onderzoek is er geen manier ontdekt waarbij er rekening kan gehouden worden met familiale verbanden. 


\section{ Neo4j}
\label{sec:Conclusie Neo4j}