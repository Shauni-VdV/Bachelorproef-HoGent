%%=============================================================================
%% Voorwoord
%%=============================================================================

\chapter*{\IfLanguageName{dutch}{Woord vooraf}{Preface}}
\label{ch:voorwoord}

%% TODO:
%% Het voorwoord is het enige deel van de bachelorproef waar je vanuit je
%% eigen standpunt (``ik-vorm'') mag schrijven. Je kan hier bv. motiveren
%% waarom jij het onderwerp wil bespreken.
%% Vergeet ook niet te bedanken wie je geholpen/gesteund/... heeft

Deze bachelorproef met als titel ``Personalized Search: Graph vs Elasticsearch`` werd geschreven in het kader van het behalen van een diploma als bachelor in de Toegepaste Informatica aan de Hogeschool Gent.

Dit onderwerp werd mij voorgesteld door MultiMinds, en sprak mij aan omdat dit ondanks de recente wetgeving rond GDPR nog steeds een actueel onderwerp is. Ik had reeds een idee in mijn hoofd  van hoe personalisatie in e-commerce toepassingen in zijn werk ging, en ik wou mij hier met plezier in verdiepen. Dit onderzoek was een leerzame ervaring voor mij, en als ik erop terugblik zou ik geen ander onderwerp gekozen hebben. 

Bij deze zou ik graag nog even enkele personen bedanken die mij geholpen en gesteund hebben bij het maken van deze bachelorproef, zonder hun zou dit onderzoek niet tot stand kunnen gekomen zijn.

Als eerste zou ik graag mijn copromotor, Nicolas Lierman van MultiMinds, bedanken voor het aanbieden van dit interessante onderwerp, alsook de technische ondersteuning en het nazien van dit werk.

Ten tweede zou ik ook graag mijn promotor, Guy Dekoning van Hogeschool Gent, bedanken voor de vlotte en zeer duidelijke feedback, en het blijven steun tonen in mijn onderzoek. 

Ten slotte wil ik ook graag mijn ouders bedanken, om mij in alle rust te laten werken ondanks dat ik hele dagen thuiszit; mijn medestudenten, die mijn rubberen eendje wouden spelen; en mijn vrienden, in het bijzonder Angelo Carly, om mij morele steun te bieden en inhoudelijke feedback te geven van het perspectief van een buitenstaander.


