\chapter{\IfLanguageName{dutch}{Stand van zaken}{State of the art}}
\label{ch:stand-van-zaken}

% Tip: Begin elk hoofdstuk met een paragraaf inleiding die beschrijft hoe
% dit hoofdstuk past binnen het geheel van de bachelorproef. Geef in het
% bijzonder aan wat de link is met het vorige en volgende hoofdstuk.

% Pas na deze inleidende paragraaf komt de eerste sectiehoofding.

\section{Personalisatie}
\label{sec:Personalisatie}

Personalisatie van webapplicaties en websites draait erom de bezoekers een op maat gemaakte ervaring aan te bieden. Dit kan op verschillende manieren toegepast worden, en kan meerdere doelen hebben. In dit hoofdstuk wordt er verder ingegaan op welke manieren personalisatie van websites wordt toegepast, en wat dit betekent voor zowel de bezoeker als het bedrijf zelf.

\subsection{Personalisatie op basis van e-mail en sociale media}
\label{subsec:Personalisatie op basis van e-mail en Social Media}

Zowat iedereen heeft wel te kampen met een overvloed aan e-mails in hun postvak van allerlei websites waar ze hun e-mailadres ooit hebben vrijgegeven. U zou denken dat deze door de meeste mensen simpelweg verwijderd worden, maar e-mailmarketing blijft een van de meest succesvolle marketingstrategieën \autocite{Dehkordi2012}. E-mailmarketing is relatief makkelijk te implementeren en vereist weinig technische investering, meestal wordt dit verwezenlijkt via systemen van derden zoals bijvoorbeeld \href{https://mailchimp.com/}{MailChimp}. Een nadeel van deze marketingvorm is dat het bedrijf continu bezig moet zijn met nieuwe inhoud te creëren voor deze e-mails, alsook op de website waar de marketingmails over gaan. 

\subsection{Personalisatie op basis van geografische locatie}
\label{subsec:Personalisatie op basis van geografische locatie}

Geografische personalisatie is het aanpassen van de website op basis van de locatie van de gebruiker. Gebruikers uit België die naar de website van een internationaal bedrijf surfen, zullen dan worden omgeleid naar een Nederlandse of Franse versie van die website.

Geografische personalisatie kan ook gebruikt worden om de inhoud van een pagina aan te passen aan de hand van de locatie van de gebruiker, of om vertalingen aan te bieden.
Een nadeel hiervan is dat mensen die op reis gaan het soms moeilijk zouden kunnen hebben om naar de juiste versie van de website te navigeren, aangezien het systeem de gebruiker zal willen omleiden naar de pagina of inhoud die voorzien is voor het land waar zij zich momenteel in bevinden. Eenzelfde probleem kan zich voordoen bij bedrijven die hun webverkeer omleiden via een ander land door middel van bijvoorbeeld een VPN. 

Geografische personalisatie is ook relatief eenvoudig te implementeren en kan een grote troef zijn op de internationale markt. 

\subsection{Personalisatie op basis van IP-adres}
\label{subsec:Personalisatie op basis van IP-adres}

Deze methode van personalisatie is wat minder opvallend, aangezien het bij de gemiddelde internetgebruiker weinig tot nooit zal voorkomen, aangezien zij het internet gebruiken via een serviceprovider zoals Telenet of Proximus. 

Deze vorm van personalisatie wordt gebruikt om zakelijke gebruikers en bedrijven te kunnen identificeren op basis van hun IP-adres. Zo kan men zien of een bezoeker bij een bepaald bedrijf werkzaam is om deze direct aan te spreken op bijvoorbeeld de homepagina.

	\includegraphics[width=\linewidth]{img/e2fcfe784532c41a644e4465f535530d}

Net zoals bij personalisatie op basis van locatie kan dit misleidende resultaten opleveren, bijvoorbeeld als de werknemer van thuis werkt of het IP-adres niet duidelijk aantoont vanuit welk bedrijf het webverkeer van de bezoeker afkomstig is. Ook voor performantie kan dit negatieve gevolgen hebben, aangezien deze vorm van personalisatie afhankelijk is van systemen van derden. 

Verder moet er ook inhoud gecreëerd worden voor elk bedrijf dat men specifiek wil aanspreken. Dit is een tijdrovend proces, maar aangezien deze vorm van personalisatie weinig voorkomt, is het wel een troef waardoor het bedrijf zich kan onderscheiden van de meerderheid en zich kan laten opvallen.

  
 \subsection{Verwante inhoud personalisatie}
 \label{subsec:Verwante inhoud personalisatie}
 
 Dit is de vorm van personalisatie die een grote meerwaarde zal leveren aan dit onderzoek. De meeste mensen hebben deze vorm al ondervonden op een webshop zoals Amazon of Bol.com. Deze vorm draait erom de gebruikers artikels aan te raden op basis van artikels of inhoud die ze al eerder bekeken hebben, alsook het gedrag van andere gebruikers.
 
De werking van het aanbevelingssysteem van Amazon is gebaseerd op enkele complexe algoritmen  \autocite{Linden2003}. Dit is natuurlijk verantwoord omdat zij op zeer grote schaal werken en veel geld hebben geïnvesteerd in de ontwikkeling van hun systeem. 

In de realiteit hoeven de technologieën voor aanbevelingen van producten niet zo complex te zijn voor gewone webshops en bedrijven, vaak is het voldoende om relaties te creëren tussen artikels en op basis van deze relaties nieuwe artikels aan te raden aan de gebruikers. 

Een voorbeeld van een relatie tussen twee artikels is de welbekende 'Anderen bekeken ook' blok die vaak zichtbaar is bij het bekijken van een detailpagina van een product. 
Een simpelere methode van dergelijke relaties is het aanbieden van verwante producten op basis van categorieën of tags. Tags zijn een manier om kenmerken van een product weer te geven die specifieker zijn dan een categorie. Een categorie kan dan 'schoenen' zijn, terwijl een tag 'lage sneakers' is. 

\section{Wat is Graph?}
\label{sec:wat is Graph?}



\section{Wat is ElasticSearch?}
\label{sec:wat is ElasticSearch?}



