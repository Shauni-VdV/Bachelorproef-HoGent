\chapter{\IfLanguageName{dutch}{Stand van zaken}{State of the art}}
\label{ch:stand-van-zaken}

% Tip: Begin elk hoofdstuk met een paragraaf inleiding die beschrijft hoe
% dit hoofdstuk past binnen het geheel van de bachelorproef. Geef in het
% bijzonder aan wat de link is met het vorige en volgende hoofdstuk.

% Pas na deze inleidende paragraaf komt de eerste sectiehoofding.

\section{Personalisatie}
\label{sec:Personalisatie}

Personalisatie van webapplicaties en websites draait erom de bezoekers een op maat gemaakte ervaring aan te bieden. Dit kan op verschillende manieren toegepast worden, en kan meerdere doelen hebben. 

In dit hoofdstuk wordt er verder ingegaan op welke manieren personalisatie van websites wordt toegepast, en wat dit betekent voor zowel de bezoeker als het bedrijf zelf.

\subsection{Personalisatie op basis van e-mail en sociale media}
\label{subsec:Personalisatie op basis van e-mail en Social Media}

Zowat iedereen heeft wel te kampen met een overvloed aan e-mails in hun postvak van allerlei websites waar ze hun e-mailadres ooit hebben vrijgegeven. Het is een courante denkwijze om er van uit te gaan dat deze  e-mails door de meeste mensen simpelweg verwijderd worden, maar e-mailmarketing blijft een van de meest succesvolle marketingstrategieën \autocite{Dehkordi2012}. 

E-mailmarketing is relatief makkelijk te implementeren en vereist weinig technische investering, meestal wordt dit verwezenlijkt via systemen van derden zoals bijvoorbeeld \href{https://mailchimp.com/}{MailChimp}. 

Een nadeel van deze marketingvorm is dat het bedrijf continu bezig moet zijn met nieuwe inhoud te creëren voor deze e-mails, ook moet de website continu geüpdatet worden met relevante informatie en pagina's die relevant zijn voor de e-mails die ze versturen.

\subsection{Personalisatie op basis van geografische locatie}
\label{subsec:Personalisatie op basis van geografische locatie}

Geografische personalisatie is het aanpassen van de website op basis van de locatie van de gebruiker. Gebruikers uit België die naar de website van een internationaal bedrijf surfen, zullen dan worden omgeleid naar een Nederlandse of Franse versie van die website.

Geografische personalisatie kan ook gebruikt worden om de inhoud van een pagina aan te passen aan de hand van de locatie van de gebruiker, of om vertalingen aan te bieden.

Een nadeel hiervan is dat mensen die op reis gaan het soms moeilijk zouden kunnen hebben om naar de juiste versie van de website te navigeren, aangezien het systeem de gebruiker zal willen omleiden naar de pagina of inhoud die voorzien is voor het land waar zij zich momenteel in bevinden. Eenzelfde probleem kan zich voordoen bij bedrijven die hun webverkeer omleiden via een ander land door middel van bijvoorbeeld een VPN. 

Een mogelijke oplossing voor dit probleem is het bijhouden van cookies. Als de geografische informatie over een gebruiker daarin wordt opgeslagen, kan deze via de cookie opgehaald worden en zal een gebruiker alsnog de website met de correcte inhoud en taal aangeboden kregen. Als de gebruiker geen cookies toelaat, of zijn cookies verwijdert, wordt deze mogelijke oplossing tenietgedaan.

Geografische personalisatie is ook relatief eenvoudig te implementeren en kan een grote troef zijn op de internationale markt. 

\subsection{Personalisatie op basis van IP-adres}
\label{subsec:Personalisatie op basis van IP-adres}

Deze methode van personalisatie is wat minder opvallend, aangezien het bij de gemiddelde internetgebruiker weinig tot nooit zal voorkomen, dit omdat zij het internet gebruiken via een serviceprovider zoals Telenet of Proximus. 

Deze vorm van personalisatie wordt gebruikt om zakelijke gebruikers en bedrijven te kunnen identificeren op basis van hun IP-adres. Zo kan men zien of een bezoeker bij een bepaald bedrijf werkzaam is om deze direct aan te spreken op bijvoorbeeld de homepagina.

	\includegraphics[width=\linewidth]{img/e2fcfe784532c41a644e4465f535530d}

Net zoals bij personalisatie op basis van locatie kan dit misleidende resultaten opleveren, bijvoorbeeld als de werknemer van thuis werkt of het IP-adres niet duidelijk aantoont vanuit welk bedrijf het webverkeer van de bezoeker afkomstig is. Ook voor performantie kan dit negatieve gevolgen hebben, aangezien deze vorm van personalisatie afhankelijk is van systemen van derden. 

Verder moet er ook inhoud gecreëerd worden voor elk bedrijf dat men specifiek wil aanspreken. Dit is een tijdrovend proces, maar aangezien deze vorm van personalisatie weinig voorkomt, is het wel een troef waardoor het bedrijf zich kan onderscheiden van de meerderheid en zich kan laten opvallen.

  
 \subsection{Verwante inhoud personalisatie}
 \label{subsec:Verwante inhoud personalisatie}
 
 Dit is de vorm van personalisatie die een grote meerwaarde zal leveren aan dit onderzoek. De meeste mensen hebben deze vorm al ondervonden op een webshop zoals Amazon of Bol.com. Deze vorm van personalisatie draait erom de gebruikers artikels of producten aan te raden op basis van items die ze al eerder bekeken hebben, alsook het gedrag van andere gebruikers.
 
De werking van het aanbevelingssysteem van Amazon is gebaseerd op enkele complexe algoritmen. \autocite{Linden2003} Dit is natuurlijk verantwoord omdat zij een gigant zijn in de e-commerce industrie. Amazon werkt op zeer grote schaal werken en heeft veel geld geïnvesteerd in de ontwikkeling van hun systeem. 

In de realiteit hoeven de technologieën voor aanbevelingen van producten niet zo complex te zijn voor gewone webshops en bedrijven, vaak is het voldoende om relaties te creëren tussen artikels en op basis van deze relaties nieuwe artikels aan te raden aan de gebruikers. 

Een voorbeeld van een relatie tussen twee artikels is de welbekende 'Anderen bekeken ook' blok die vaak zichtbaar is bij het bekijken van een detailpagina van een product. 

Een simpelere methode van dergelijke relaties is het aanbieden van verwante producten op basis van categorieën of tags. Tags zijn een manier om kenmerken van een product weer te geven die specifieker zijn dan een categorie. Een categorie kan dan 'schoenen' zijn, terwijl een tag 'lage sneakers' is. 

Dit is een klassiek voorbeeld van de probleemstelling van dit onderzoek. De zoekfunctie is dus beperkt tot de productcatalogus en de informatie die beschikbaar is over deze producten. De website maakt dus geen gebruik van persoonlijke informatie om deze zoekresultaten te personaliseren.

Over de mogelijkheden van het personaliseren van zoekresultaten gaan we dieper in in het volgende hoofdstuk van deze literatuurstudie.

\section{Recommender Systems}
\label{sec:Recommender Systems}
Recommender Systems \autocite{Resnick1997} zijn aanbevelingen vanuit het systeem die rekening houden met de beschikbare informatie van gebruikers en hun voorkeuren om zo een filter te plaatsen op de informatie die weergegeven wordt. Verder zullen we deze benoemen aan de hand  van hun Nederlandse naam 'aanbevelingssystemen'.

Aanbevelingssystemen worden vooral gebruikt in een e-commerce toepassingen waar een zeer groot en verscheiden aanbod aan producten is, en het al vaak lastig wordt om precieze aanbevelingen aan de klant te geven. Hierbij wordt allerlei informatie van een gebruiker verzameld, zoals historische aankopen, items op het verlanglijstje, items waar de gebruiker op geklikt heeft, etc.

In dit onderdeel zullen we kort wat dieper ingaan op de werking van dergelijke aanbevelingssystemen en de achterliggende algoritmen.

\subsection{Haalbaarheid}
\label{sec:Haalbaarheid}

Een degelijk systeem biedt een grote meerwaarde voor een bedrijf, zo heeft Netflix een competitie gehouden die 1 miljoen dollar bood aan degene die een aanbevelingssysteem kon maken dat 10\% beter presteerde dan hun bestaande systeem. 

Deze wedstrijd liep van oktober 2006 tot minstens oktober 2011. De wedstrijd werd de \cite{NetflixPrize} genoemd, Netflix stelde hiervoor een dataset beschikbaar. Enkele groepen hebben het doel behaald, maar het algoritme van de winnende groep is uiteindelijk nooit in productie gebracht, omdat de mogelijke opbrengst niet kon opwegen tegen de extra gevraagde rekenkracht. 

Een aanbevelingssysteem moet dus niet enkel de juiste waarden kunnen aangeven, het moet ook realistisch en haalbaar zijn qua rekenkracht en extra bijkomende kosten voor onderhoud. Het ontwerpen van dergelijk systeem is dus geen eenvoudige klus.

\subsection{Algoritmen voor aanbevelingssystemen}
\label{sec:Algoritmen voor aanbevelingssystemen}

We spreken over twee grote categorieën in algoritmen van aanbevelingssystemen: 'collaborative filtering'-algoritmen en 'content based'-algoritmen.  \autocite{Adamovicius2005} Door de evolutie en groeiende ontwikkeling van beter presterende systemen zijn er ondertussen ook enkele algoritmen die niet echt binnen een van deze twee koepels vallen. In de onderstaande figuur wordt een simpele representatie gegeven van de werking van deze soorten algoritmes.

	\includegraphics[width=\linewidth]{img/Content-based-filtering-and-Collaborative-filtering-recommendation}
	
\subsubsection{Collaborative Filtering}
\label{sec:Collaborative Filtering}

Het kernidee bij Collaborative Filtering \autocite{Schafera} is om aanbevelingen te maken gebaseerd op de voorkeuren van een andere gebruiker met een gelijkaardig gedrag. Zoals de figuur hierboven aantoont dat als gebruiker 1 en 2 hetzelfde artikel gelezen hebben, gebruiker 2 een artikel als aanbeveling zal krijgen dat gelezen werd door gebruiker 1. 
Hetzelfde idee kan toegepast worden op allerlei interacties van de gebruikers met een website zoals likes, shares, verlanglijstjes, etc.

In de praktijk geeft deze techniek zeer goede resultaten, maar zoals verwacht brengt deze techniek ook enkele problemen met zich mee. Het meest merkwaardige probleem is een zogenaamde cold start, dit komt onder andere voor bij de eerste interactie van een nieuwe gebruiker met een applicatie die aanbevelingen biedt. Het algoritme heeft dan onvoldoende informatie om een correcte en nuttige aanbeveling op te leveren aan de gebruiker. Een andere oorzaak kan zijn dat er een nieuw product wordt toegevoegd aan het systeem, er kan dan nog niet geweten zijn welk type gebruiker hierin geïnteresseerd zou kunnen zijn. 

Een andere factor voor het succes van een Collaborative Filtering algoritme is het aantal gebruikers van een systeem, met andere woorden, hoe meer gebruikers er zijn, hoe correcter de aanbevelingen aan een specifieke gebruiker zal zijn. \autocite{Sarwar2001}. Een gebruiker met ongewone interesses zal logischerwijs in dergelijk klein systeem weinig gelijkaardige gebruikers hebben, en zal dus ook geen optimale aanbevelingen krijgen.


Een groot voordeel van Collaborative Filtering is dat er absoluut geen kennis hoeft te zijn van de toepassing van het systeem, de aanbevelingen worden gegenereerd op basis van het gedrag van de gebruikers en zijn interesses. De producten of hun attributen moeten dus niet gekend zijn om aanbevelingen te kunnen geven, dat maakt het eenvoudiger om een aanbevelingssysteem met de techniek van Collaborative Filtering te implementeren.

\subsubsection{Content Based}
\label{sec:Content Based}

Content Based aanbevelingssystemen \autocite{Lops2011} maken, in tegenstelling tot Collaborative Filtering, wel gebruik van de specifieke producten binnen het systeem en hun attributen. Op basis van deze attributen en de interesse van de gebruiker daarin, wordt per gebruiker een profiel opgezet, elk attribuut krijgt dan een score toegekend, een hogere score betekent grotere interesse. Een attribuut van een product kan dan bijvoorbeeld 'schoenen' of 'PS4 games' zijn, of zelfs een filmgenre.

Een probleem van Content Based is, net zoals bij Collaborative Filtering, het cold start probleem. Als een nieuwe gebruiker het systeem gebruikt, is er voor deze gebruiker nog geen profiel opgesteld en kunnen er ook geen anbevelingen gemaakt worden.

Een ander probleem van Content Based wordt overspecialisatie genoemd, dit treedt op wanneer het systeem eigenlijk té accurate aanbevelingen doet. Het gevolg hiervan is dat slechts enkele producten voldoen aan de verwachtingen van het systeem, waardoor er geen nieuwe aanbevelingen aan de gebruiker naar voor gebracht worden, en de gebruiker enkel producten zal zien die hij reeds bekeken heeft. 

Het andere probleem dat optreedt bij Collaborative Filtering, namelijk dat bij het toevoegen van nieuwe producten niet geweten kan zijn welke gebruikers hierin geïnteresseerd zouden zijn, is niet van toepassing bij Content Based. Het systeem maakt gebruik van de attributen van producten, dus nieuwe producten kunnen meteen belanden in de aanbevelingen van gebruikers die reeds interesse getoond hebben in andere producten met die attributen. Ook het aantal gebruikers binnen een systeem vormt om dezelfde reden geen probleem bij Content Based aanbevelingssystemen. 

Een grote boosdoener bij Content Based kan zijn dat producten slecht gelabeld zijn, en hun attributen onvoldoende passen bij wat het product effectief is. Hierdoor kan het systeem deze producten niet goed vergelijken met andere. Dit is vooral een probleem wanneer de attributen van de producten van verschillende bronnen afkomstig zijn, of manueel slecht opgesteld zijn.

\subsubsection{Hybrides}
\label{sec:Hybrides}

Beide van de voorgenoemde technieken hebben elk hun eigen voor- en nadelen, alsook sterke en zwakke punten. Content Based heeft te kampen met overspecialisatie, maar is wel in staat om nieuwe producten meteen aan te bevelen aan de gebruikers. Collaborative Filtering heeft moeite met het aanbevelen van nieuwe producten en een cold start, maar heeft geen problemen in de aard van overspecialisatie.

De logische redenering is dan natuurlijk om deze twee soorten systemen te gaan combineren, kwestie van het beste van twee werelden te proberen bekomen. Dit worden hybride aanbevelingssystemen \autocite{Cano2017} genoemd.

De manier van het opbouwen van een hybride systeem kan zijn dat beide methoden afzonderlijk worden uitgevoerd, en hun resultaten op het einde samengebundeld worden. Dit is de meest eenvoudige implementatie.

Een andere manier van combineren kan zijn door de informatie binnen een Collaborative Filtering systeem aan te vullen met informatie uit de gebruikersprofielen. Hierdoor wordt de gelijkaardigheid van twee producten bepaald door zowel de inhoud en welke soorten typische gebruikers deze producten bekijken, kopen, leuk vinden, etc.
De informatie van deze gebruikers kan dan bijvoorbeeld leeftijdsgroep, woonplaats, gezinssamenstelling, etc. voorstellen.


\section{Wat is Graph?}
\label{sec:wat is Graph?}

In de context van deze bachelorproef zal er met 'Graph' steeds verwezen worden naar een Graph databank. In dit onderdeel zullen we wat dieper ingaan op wat dit soort databank precies inhoudt om een volledig begrijpen van de precieze werking te verzekeren. 

\subsection{Graph}
\label{sec:Graph}

Graph is de databankstructuur die verder in dit onderzoek gebruikt zal worden. Deze structuur maakt gebruik van een wiskundige graaf om data op te slaan. Een graaf bestaat uit een aantal knopen (genaamd nodes) die al dan niet verbonden zijn. 

\includegraphics[width=\linewidth]{img/Customer-Order-Example-Graph.png}

Relaties hebben de prioriteit bij een graph databank, dit betekend dat relaties tussen knopen gepersisteerd worden in de databank, en niet enkel tijdelijk berekend voor een enkele query, dit is de reden waarom er geen complexe JOIN query's of foreign keys nodig zijn.

Graph databanken hebben dus een voordeel voor functionaliteiten zoals netwerken, aanbevelingen, etc. maar ook fraudedetectie. 

\subsubsection{Voordelen}
\label{subsec:Voordelen Graph}

Een groot voordeel van de graafstructuur die deze databanken gebruiken is dat deze sneller zijn dan relationele databanken omdat ze op zeer korte tijd naar een bepaalde node kunnen verwijzen, waarbij we bij een relationele databank een JOIN zouden moeten gebruiken. 

Graph databanken zijn ook zeer flexibel in gebruik. Bij het veranderen van een model kan er steeds verder gebouwd worden op het bestaande model, zonder dat hierdoor reeds verwezenlijkte functionaliteiten verloren zou kunnen gaan.
Dit is ook een voordeel bij het opstarten, er moet niet op voorhand grondig gediscussieerd worden over hoe het model er uiteindelijk uit zal moeten zien.

Deze troeven zijn niet te vinden bij traditionele databankstructuren die met tabellen werken, daarbij moet er alvorens het opzetten van de databank al een zekerheid zijn van hoe het model eruit zal zien, want het model zal de mogelijke functionaliteiten bepalen. Als er dan achteraf nog veranderingen nodig zijn, zal heel het model herzien moeten worden.

Nog een belangrijk voordeel van graph databanken is dat hiermee een \textit{Knowledge Graph} kan opgebouwd worden, hier wordt in een verder hoofdstuk van de literatuurstudie dieper op ingegaan.

\subsubsection{Nadelen}
\label{subsec: Nadelen Graph}

Graph databanken hebben niet enkel voordelen, deze zijn vooral afhankelijk van de verwachtingen van de gebruikers en wat zij nodig hebben om hun doeleinden te bereiken. Graph databanken bestaan niet om relationele databanken te vervangen, zij bieden slechts een oplossing voor de beperkingen waar relationele databanken mee kampen.

Graph databanken zijn geen goede keuze wanneer er data moet opgeslagen worden waarbij geen verbanden of connecties zijn tussen deze data, hier blinkt een relationele databank dan weer in uit. 

Hetzelfde geldt voor wanneer de databank slechts als opslag gebruikt zal worden, of er slechts simpele query's nodig zullen zijn die in een relationele databank zonder JOIN statement kunnen gebeuren.Dit is niet zozeer een nadeel, maar benadrukt wel dat graph databanken geen universele \textit{one size fits all} oplossing zijn. Graph kan dit, maar is hiervoor niet geoptimaliseerd.

Een voordeel dat eigenlijk als nadeel kan aanzien worden is dat de data niet consistent hoeft te zijn, het model dat opgebouwd wordt is niet hetzelfde als een schema bij relationele databanken. Als er een databank van personen wordt opgesteld, waar zij een connectie hebben met een bedrijf als 'werknemer', maar ook met een dier als 'huisdier', gaat een graph databank het toestaan om een relatie van het huisdier als 'werknemer' te leggen naar het bedrijf. Dit is natuurlijk niet wenselijk.

De voor- en nadelen van graph databanken worden nog even kort opgesomd in volgende tabel:

\begin{center}
	\begin{tabularx}{\textwidth}{|X|X|}
		\hline
		\textbf{Voordelen} & \textbf{Nadelen}  \\ 
		\hline
		 - Sneller dan relationele databanken bij het queryen met meerdere tabellen & -Gaat slecht om met data waar weinig of geen connecties zijn \\ 
		- Flexibel & - Data is niet gebonden aan een schema \\
		- Makkelijk te begrijpen structuur & - Niet geoptimaliseerd voor simpele query's waar in een relationele databank geen JOIN nodig is \\ 
		- Opbouwen van Knowledge Graph & \\
		\hline
	\end{tabularx}
\end{center}

\subsection{Knowledge Graphs}
\label{sec:Knowledge Graphs}

Google startte in 2012 een initiatief genaamd \textit{Knowledge Graph}. Deze graaf wordt door Google gebruikt om de resultaten van de Google zoekmachine aan te vullen met informatie uit verscheidene bronnen. De informatie uit deze Knowledge Graph is terug te vinden in een kader rechts van de gewone resultaten, zoals te zien is in volgende figuur: 

\includegraphics[width=\linewidth]{img/Google_Knowledge_Graph.jpeg}


\subsection{Neo4j}
\label{sec:Neo4j}

Er bestaan verschillende platformen voor SQL, alsook verschillende platformen voor de hierboven beschreven NoSQL databanken. Neo4j is een van de meer bekende platformen voor graph databanken, dit platform zal gebruikt worden in dit onderzoek.

Walmart maakt gebruik van Neo4j om de aanbevelingen voor hun klanten op hun online webservices te optimaliseren \autocite{neo4jWalmart2014}. Zij gebruiken dit omdat graph databanken zeer snel over een gebruiker zijn koophistorie kunnen traverseren, en ook direct nieuwe mogelijke interesses kunnen halen uit het gedrag van de gebruiker. Daarmee wordt bedoeld dat er in real-time nieuwe connecties worden gelegd tussen de gebruiker en de producten, en hij de nieuwe aanbevelingen meteen zal zien, en niet enkele dagen of uren later. Er wordt dus historische data gematcht met real-time data, hier blinkt Neo4j in uit. 

Neo4j maakt dus gebruik van wiskundige grafen om data weer te geven samen met hun onderlinge relaties. Een graaf kan gericht of ongericht zijn, ongericht wil zeggen dat er geen richting is waarin de relaties lopen, dus deze zijn onderling uitwisselbaar. Een gerichte graaf is dan een graaf waarbij de relaties in een specifieke richting lopen, zoals volgers op Twitter: Persoon A volgt persoon B, maar persoon B volgt niet persoon A.

\section{Wat is ElasticSearch?}
\label{sec:wat is ElasticSearch?}

Elasticsearch op zichzelf is eigenlijk een zoekmachine die data kan analyseren, of dit nu nummers, tekst, gestructureerd of ongestructureerd is. Elasticsearch is een component van de Elastic stack, ook wel ELK-Stack genoemd. Dit staat voor Elasticsearch, Logstash en Kibana. Logstash wordt gebruikt om data te verwerken uit meerdere bronnen en te versturen naar Elasticsearch. Kibana is een tool om de gebruikers een visueel beeld te geven van de data in de vorm van grafieken of tabellen.

In Elasticsearch is het mogelijk gewichten toe te kennen aan de resultaten van een zoekopdracht, zo kunnen meer relevante resultaten hoger in de lijst staan, dit is een belangrijke factor in dit onderzoek, namelijk of deze technologie gebruikt kan worden in E-Commerce toepassingen. In het artikel van \cite{Vavliakis2019} wordt beweerd dat het systeem dat zij implementeerden op een performante manier de gewenste resultaten gaf, alsook dat dit een oplossing kan zijn voor real-time zoekopdrachten in commerciële toepassingen.


\section{Filter Bubble}
\label{sec:Filter Bubble}

Een filter bubble, ook wel informatieluchtbel genaamd, is een gevolg van het personaliseren van zoekopdrachten. Personalisatie is het proberen bepalen welke zaken een gebruiker zou willen zien, aan de hand van een algoritme. Een filter bubble betekent dat door deze personalisatie, een bijvoorbeeld geen producten zal zien die hem niet interesseren, in dit geval is dat een voordeel. Als we kijken in de context van controversiële onderwerpen, zorgt deze bubbel ervoor dat de gebruiker eigenlijk geen informatie zal zien die niet bij zijn standpunt past, de gebruikers worden dus afgesloten in luchtbel bestaande uit enkel hun visie of standpunt.  \autocite{Pariser2011}

De relevantie van dit fenomeen voor dit onderzoek valt op het feit dat bij de Content Based algoritmen die eerder besproken werden, we met een gelijkaardig probleem te maken hadden. Bij deze aanbevelingssystemen kan er overspecialisatie optreden, waardoor de gebruiker enkel producten aanbevolen krijgt die reeds gezien zijn, of té gelijkaardig zijn, en er dus geen nieuwe verrassende producten aangeboden worden.

\section{GDPR}
\label{sec:GDPR}


