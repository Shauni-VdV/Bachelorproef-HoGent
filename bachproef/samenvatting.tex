%%=============================================================================
%% Samenvatting
%%=============================================================================

% TODO: De "abstract" of samenvatting is een kernachtige (~ 1 blz. voor een
% thesis) synthese van het document.
%
% Deze aspecten moeten zeker aan bod komen:
% - Context: waarom is dit werk belangrijk?
% - Nood: waarom moest dit onderzocht worden?
% - Taak: wat heb je precies gedaan?
% - Object: wat staat in dit document geschreven?
% - Resultaat: wat was het resultaat?
% - Conclusie: wat is/zijn de belangrijkste conclusie(s)?
% - Perspectief: blijven er nog vragen open die in de toekomst nog kunnen
%    onderzocht worden? Wat is een mogelijk vervolg voor jouw onderzoek?
%
% LET OP! Een samenvatting is GEEN voorwoord!

%%---------- Nederlandse samenvatting -----------------------------------------
%
% TODO: Als je je bachelorproef in het Engels schrijft, moet je eerst een
% Nederlandse samenvatting invoegen. Haal daarvoor onderstaande code uit
% commentaar.
% Wie zijn bachelorproef in het Nederlands schrijft, kan dit negeren, de inhoud
% wordt niet in het document ingevoegd.

\IfLanguageName{english}{%
\selectlanguage{dutch}
\chapter*{Samenvatting}
\lipsum[1-4]
\selectlanguage{english}
}{}

%%---------- Samenvatting -----------------------------------------------------
% De samenvatting in de hoofdtaal van het document

\chapter*{\IfLanguageName{dutch}{Samenvatting}{Abstract}}

In samenwerking met MultiMinds werd besloten onderzoek te voeren naar de haalbaarheid van het gebruik van twee vooraf bepaalde opties voor het implementeren van een gepersonaliseerde zoekfunctie bij e-commerce toepassingen, de  opties die gekozen werden door de opdrachtgever zijn Neo4j en Elasticsearch.  

De reden voor dit onderzoek is dat zo goed als alle gekende webshops gebruik maken van een zoekfunctie, maar dat deze vaak beperkt blijft tot de productcatalogus. In dit onderzoek wordt dus nagegaan of het mogelijk is deze zoekfunctie uit te breiden met data van gebruikers om zo een meer gepersonaliseerd resultaat te kunnen opleveren.

Enkele voorbeelden van dergelijke gebruikersdata zijn geslacht, leeftijd, locatie, etc., maar ook bijvoorbeeld familiale verbanden. Zo is het bijvoorbeeld wenselijk dat in het geval van twee samenwonenden, Persoon A en persoon B, er merken worden aanbevolen aan Persoon A waarvan Persoon B regelmatig producten koopt.

In eerste instantie werd er een literatuurstudie gedaan naar de reeds bestaande vormen van personalisatie, al dan niet binnen e-commerce toepassingen als marketingvorm.

Ten tweede werden de verschillende types van aanbevelingssystemen onderzocht. Daarbij werden twee belangrijke families gevonden: Collaborative Filtering-algoritmen, en Content Based-algoritmen. Voor de implementatie van dergelijke algoritmen zijn ook hybride vormen mogelijk, deze hybride vormen werden gebruikt in dit onderzoek.

Ten derde werd er ook een literatuuronderzoek gedaan naar de twee platformen, Neo4j en Elasticsearch. In dit deel van het onderzoek werd bekeken of deze platformen de mogelijkheden hadden om een aanbevelingssysteem te implementeren dat voldoende accurate resultaten zou opleveren. Uit dit onderzoek bleek dat Neo4j sterk uitblinkt in het gebruiken van data over de relaties tussen twee personen, waar Elasticsearch eerder uitblinkt in het zeer snel opleveren van resultaten die aan bepaalde criteria voldoen.

Als vierde puntje in het onderzoek werd ook verder ingegaan op een nadeel dat het personaliseren van e-commerce toepassingen met zich mee kan brengen, namelijk een 'Filter Bubble', waarbij gebruikers enkel nog resultaten te zien krijgen die volledig passen bij hun profiel. Bijgevolg zullen gebruikers weinig of zelfs geen resultaten mogen verwachten die door hun nog niet gezien werden.

Als laatste deel van de literatuurstudie werd bekeken of de implementatie van een systeem dat op deze manier gebruik maakt van persoonlijke data en gegevens wel als legaal aanschouwd kan worden onder de wetgeving van de GDPR. Onderzoek wees uit dat het systeem hier geen problemen van zou mogen ondervinden.

De resultaten van dit onderzoek wezen uit dat beide platformen hun eigen voor- en nadelen met zich meebrachten, verwijzend naar de verschillende functionaliteiten die onze zoekfunctie vereist. Een belangrijke factor bij het maken van de keuze is ook de eenvoud van de implementatie, waarbij al snel bleek dat het zeer moeilijk is om de sterktes van Neo4j te repliceren in een basisversie van Elasticsearch, of omgekeerd. Voor een bedrijf wordt aangeraden om voor Elasticsearch te kiezen met een Enterprise licentie, zodat de Graph API functionaliteit gebruikt kan worden.

De resultaten zetten aan tot verder onderzoek naar de mogelijkheid om deze twee systemen parallel te gebruiken, en de resultaten ervan te combineren.  Zo kan Neo4j gebruikt worden als een 'Knowledge Graph', die dient als aanvulling voor de data die Elasticsearch tot zijn beschikking heeft. 


