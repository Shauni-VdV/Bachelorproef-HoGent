%==============================================================================
% Sjabloon onderzoeksvoorstel bachelorproef
%==============================================================================
% Gebaseerd op LaTeX-sjabloon ‘Stylish Article’ (zie voorstel.cls)
% Auteur: Jens Buysse, Bert Van Vreckem
%
% Compileren in TeXstudio:
%
% - Zorg dat Biber de bibliografie compileert (en niet Biblatex)
%   Options > Configure > Build > Default Bibliography Tool: "txs:///biber"
% - F5 om te compileren en het resultaat te bekijken.
% - Als de bibliografie niet zichtbaar is, probeer dan F5 - F8 - F5
%   Met F8 compileer je de bibliografie apart.
%
% Als je JabRef gebruikt voor het bijhouden van de bibliografie, zorg dan
% dat je in ``biblatex''-modus opslaat: File > Switch to BibLaTeX mode.

\documentclass{voorstel}

\usepackage{lipsum}

%------------------------------------------------------------------------------
% Metadata over het voorstel
%------------------------------------------------------------------------------

%---------- Titel & auteur ----------------------------------------------------

% TODO: geef werktitel van je eigen voorstel op
\PaperTitle{Personalized search: Graph vs. OLAP Database}
\PaperType{Onderzoeksvoorstel Bachelorproef 2019-2020} % Type document

% TODO: vul je eigen naam in als auteur, geef ook je emailadres mee!
\Authors{Shauni Van de Velde\textsuperscript{1}} % Authors
\CoPromotor{Nicolas Lierman\textsuperscript{2} (Multiminds)}
\affiliation{\textbf{Contact:}
  \textsuperscript{1} \href{mailto:shauni.vandevelde@student.hogent.be}{shauni.vandevelde@student.hogent.be};
  \textsuperscript{2} \href{mailto:nicolas.lierman@multiminds.eu}{nicolas.lierman@multiminds.eu};
}

%---------- Abstract ----------------------------------------------------------

\Abstract{ \textbf{Context} De zoekfunctie van de meeste e-commerce websites en platformen is beperkt tot de productcatalogus. Rekeninghoudend met de algemene trend rond personalisering van de customer experience zou het ook aangewezen zijn om ook de search op de e-commerce site te personaliseren. In dit onderzoek wordt bekeken welke databasetechnologie hier het meest geschikt voor is.; \textbf{Nood} Om de user experience te garanderen, is het belangrijk dat de meest performante technologie gekozen wordt, hiermee beperken we laadtijden.; \textbf{Taak} In dit onderzoek worden twee databanktechnologieën vergeleken, namelijk een OLAP-databank en een Graph databank. Het besluit hiervan zal aantonen welke technologie de beste keuze is voor Personalized Search \textbf{Object} In dit document staat het hele proces, de resultaten en de analyse beschreven; \textbf{Resultaat} Er wordt verwacht een duidelijk antwoord te kunnen bieden op de vraag over welke technologie het meest gepast is voor Personalized Search \textbf{Conclusie} Er zal ofwel een duidelijk antwoord zijn, dit wil zeggen dat één van de technologieën beter presteert dan de andere op elk gebied, ofwel zal er moeten afgewogen worden wat de gebruiker het belangrijkst vindt; \textbf{Perspectief} Toekomstgewijs is dit een goede investering van tijd en middelen. Dit is een zeer courant onderwerp dat voor vele e-commerce websites van belang is.


}

%---------- Onderzoeksdomein en sleutelwoorden --------------------------------
% TODO: Sleutelwoorden:
%
% Het eerste sleutelwoord beschrijft het onderzoeksdomein. Je kan kiezen uit
% deze lijst:
%
% - Mobiele applicatieontwikkeling
% - Webapplicatieontwikkeling
% - Applicatieontwikkeling (andere)
% - Systeembeheer
% - Netwerkbeheer
% - Mainframe
% - E-business
% - Databanken en big data
% - Machineleertechnieken en kunstmatige intelligentie
% - Andere (specifieer)
%
% De andere sleutelwoorden zijn vrij te kiezen

\Keywords{Databanken en big data. Graph --- OLAP --- Personalized Search} % Keywords
\newcommand{\keywordname}{Sleutelwoorden} % Defines the keywords heading name

%---------- Titel, inhoud -----------------------------------------------------

\begin{document}

\flushbottom % Makes all text pages the same height
\maketitle % Print the title and abstract box
\tableofcontents % Print the contents section
\thispagestyle{empty} % Removes page numbering from the first page

%------------------------------------------------------------------------------
% Hoofdtekst
%------------------------------------------------------------------------------

% De hoofdtekst van het voorstel zit in een apart bestand, zodat het makkelijk
% kan opgenomen worden in de bijlagen van de bachelorproef zelf.
%---------- Inleiding ---------------------------------------------------------

\section{Introductie} % The \section*{} command stops section numbering
\label{sec:introductie}

De meeste e-commerce online platformen hebben reeds een zoekfunctie ingebouwd, maar deze is vaak beperkt tot enkel de productcatalogus. Rekening houdend met de algemene trend rond personalisering van de customer experience zou het ook aangewezen zijn om ook de search op de e-commerce site te personaliseren. Dit zou resulteren in een betere experience voor de klant en een hogere conversie voor het bedrijf. In deze bachelorproef wordt onderzocht in hoeverre de persoonlijke data van een specifieke gebruiker en contextuele data kan ingeschakeld en gecombineerd worden met de data uit de productcatalogus, om zo persoonlijke en relevante zoekresultaten te genereren. 
Met deze persoonlijke data wordt bedoeld de historische aankoopdata van de persoon zelf, maar ook bepaalde demografische data zoals geslacht, leeftijd, gezinssamenstelling, etc. De onderzoeksvraag kan dus als volgt geformuleerd worden: Welk databankmodel presteert het best om gepersonaliseerde zoekresultaten te bekomen?


%---------- Stand van zaken ---------------------------------------------------

\section{State-of-the-art}
\label{sec:state-of-the-art}

Personalized Search \autocite{Pitkow2002} verwijst naar het zoeken op het web waarbij de resultaten afhankelijk zijn van de interesses en voorkeuren van de gebruiker die verder gaan dan de query zelf. 

Personalisatie in e-commerce toepassingen biedt een groot voordeel aan bedrijven. De loyaliteit van klanten wordt veel sterker als deze gebruik maken van gepersonaliseerde features.\autocite{Telang2005} Een zoekfunctie is een voorbeeld van zo een feature.
Recommender Systems \autocite{Resnick1997} zijn aanbevelingen vanuit het systeem die rekening houden met de beschikbare informatie van gebruikers en hun voorkeuren om zo een filter te plaatsen op de informatie die weergegeven wordt. 

De studie van \textcite{Diehl2003} onderzocht het effect van gepersonaliseerde zoekresultaten op de kwaliteit van keuzes die klanten maken, en vond een positieve correlatie. De studie ontdekte dat het verlagen van search cost \autocite{Smith1999} leidde tot minder kwaliteitsvolle keuzes. De reden daarvoor is dat klanten slechtere beslissingen maken als de search cost lager ligt omdat zij minder ideale opties aangeboden krijgen. Personalized Search en Recommender System zorgen voor een enorme verbetering in de kwaliteit van de keuzes die de klant maakt, en verminderen het aantal producten die deze klant bekijkt alvorens hij/zij gevonden heeft wat hij/zij nodig heeft.

Een gevolg van gepersonaliseerde zoekopdrachten is dat we een Filter Bubble \autocite{Pariser2011} creëren. Het verlaagt de kans dat nieuwe informatie gevonden wordt doordat de resultaten van een zoekopdracht partijdig zijn en eerder wijzen naar dingen die de gebruiker reeds gezien heeft. Dit concept wordt een Filter Bubble genoemd omdat gebruikers eigenlijk geïsoleerd worden in hun eigen wereldje, waar ze enkel de informatie te zien krijgen die ze willen zien. Als we deze gebruikers met hun bubbels in groepen opdelen, verkrijgen we wel het probleem dat deze een vertekend beeld op de realiteit krijgen, zij krijgen bijvoorbeeld in het nieuws slechts het deel te zien dat voor hun interessant is. Als we deze lijn doortrekken naar de klanten van e-commerce websites, zullen zij ook slechts de merken te zien krijgen waar hun voorkeur naar uit gaat. Hierdoor verminder je de kans dat ze een nieuw merk ontdekken of een ander product uitproberen. Als we terug refereren naar de studie van \textcite{Diehl2003}, dan kunnen we afleiden dat dit een positief effect zal hebben op de klanttevredenheid.

% Voor literatuurverwijzingen zijn er twee belangrijke commando's:
% \autocite{KEY} => (Auteur, jaartal) Gebruik dit als de naam van de auteur
%   geen onderdeel is van de zin.
% \textcite{KEY} => Auteur (jaartal)  Gebruik dit als de auteursnaam wel een
%   functie heeft in de zin (bv. ``Uit onderzoek door Doll & Hill (1954) bleek
%   ...'')


%---------- Methodologie ------------------------------------------------------
\section{Methodologie}
\label{sec:methodologie}

Om de onderzoeksvraag te beantwoorden wordt er een simpele webapplicatie opgezet waar een zoekterm kan ingevoerd worden. Deze webapplicatie zal louter gebruikt worden om met behulp van een tekstveld een query op een databank uit te voeren. Ook worden er twee databankmodellen ontworpen, een model dat gebruik maakt van Neo4j (Graph databank platform), en een model dat gebruik maakt van Elasticsearch (OLAP databank platform) en Kibana om de resultaten te visualiseren. Graph is een databankstructuur van het type NoSQL, dit wil algemeen zeggen dat ze geen gebruik maken van SQL. Bij Graph databanken worden gegevens voorgesteld door een geheel van entiteiten en verbindingen, alsook vrije relaties tussen deze entiteiten. Kortom is dit, zoals de naam al laat vermoeden, een graaf. OLAP is de afkorting voor Online Analytical Processing, dit is een technologie die geoptimaliseerd is voor het uitvoeren van query's en rapporten in plaats van transacties. Deze zullen elk met hun eigen API communiceren, en in beide modellen wordt dezelfde gebruiker- en productdata ingevoerd. Bij het opstellen van de modellen wordt mogelijk al duidelijk of één van de modellen niet in staat zal zijn om dezelfde functionaliteiten te hebben als het andere, en dan zal moeten afgewogen worden of de voordelen van het ene model opwegen tegenover het andere model. Als beide modellen dezelfde functionaliteit kunnen bereiken, wordt er bekeken welke het meest performante is. Mogelijks zou het ook haalbaar zijn om beide manieren te combineren op voorwaarde dat de responstijd binnen de acceptabele norm valt.


%---------- Verwachte resultaten ----------------------------------------------
\section{Verwachte resultaten}
\label{sec:verwachte_resultaten}

Er wordt verwacht dat er ofwel een duidelijk verschil merkbaar is in performantie tussen de twee verschillende modellen. Mogelijk is dat één van de twee modellen totaal niet haalbaar is om efficiënt een link mee te leggen tussen bijvoorbeeld familieleden, in dit geval bekijken we of het mogelijk is deze twee modellen samen uit te voeren, als dit een resultaat biedt dat binnen de norm valt qua performantie, dan is dit ook een mogelijke oplossing.
 Het kan zich ook voordoen dat beide modellen vrij performant en efficiënt de query's kunnen verwerken, in dit geval worden de voor- en nadelen alsook de moeilijkheidsgraad van implementatie afgewogen.


%---------- Verwachte conclusies ----------------------------------------------
\section{Verwachte conclusies}
\label{sec:verwachte_conclusies}

Er wordt verwacht dat het Graph model makkelijker te implementeren zal zijn en beter zal presteren als we rekening willen houden met het aankoopgedrag van vrienden en familie. Indien dit ook mogelijk is bij een OLAP-model, verwachten we dat Graph nog steeds beter zal presteren. Indien we hier geen rekening mee houden zal het OLAP-model beter presteren, aangezien dit model aangepast is aan grote hoeveelheden data waar complexe zoekopdrachten op kunnen uitgevoerd worden.




%------------------------------------------------------------------------------
% Referentielijst
%------------------------------------------------------------------------------
% TODO: de gerefereerde werken moeten in BibTeX-bestand ``voorstel.bib''
% voorkomen. Gebruik JabRef om je bibliografie bij te houden en vergeet niet
% om compatibiliteit met Biber/BibLaTeX aan te zetten (File > Switch to
% BibLaTeX mode)

\phantomsection
\printbibliography[heading=bibintoc]

\end{document}
